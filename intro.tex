\chapter{演进\statusgreen}\label{chap:vision}



本章给出了关于Swarm的动机和演进的背景信息以及它的愿景。\ref{sec:historical_context}对万维网进行了历史分析,重点分析了万维网是如何成为今天这个样子的。
\ref{sec:fair-data}介绍了这一概念,并解释了数据主权、共有信息和\gloss{fair data economy}的重要性。它讨论了一个自主社会需要的基础设施,以便能够共同地托管、移动和处理数据。
最后,\ref{sec:vision}概述了愿景背后的价值,阐述了技术的要求,并制定设计原则以指导我们展现Swarm。

\section{历史背景\statusgreen}\label{sec:historical_context}
\green{}
总体而言,互联网——尤其是万维网(\gloss{WWW})——极大地降低了信息传播的成本,将出版商的权力放在每个用户的指尖,但这些成本仍然不是零,它们的分配严重影响着谁可以发布什么内容,谁将消费它。

为了理解我们试图解决的问题,回顾一下\gloss{WWW}的发展历史是有益的。

\subsection{Web 1.0 \statusgreen}\label{sec:web_1}

在\gloss{Web 1.0}时代,为了让全世界都能访问你的内容,你通常会启动一个网络服务器,或者使用一些免费或便宜的网络主机空间来上传你的内容,然后通过一组HTML页面导航。如果你的内容不受欢迎,你仍然需要维护服务器或支付主机费用来保持它的可访问性,但真正的灾难发生时,由于这样或那样的原因,它变得受欢迎(例如,你被“slashdotted")。此时,您的流量账单在您的服务器在负载下崩溃之前暴涨,或者您的主机提供商将您的带宽限制到使您的内容对您的大多数听众基本上不可用的地步。如果你想要保持受欢迎,你必须投资于用胖管道连接的高可用性集Swarm;你的成本随着你的受欢迎程度一起增长,却没有任何明显的方法来弥补它们。几乎没有什么可行的方法能够让你的观众直接分担随之而来的经济负担。

当时有一个共识:在早期的网络变革中,维有\gloss{ISP}可以拯救我们,ISP卷入了关于提供者和消费者位置的争论,哪一个ISP能从其他的网络赚钱。事实上,当TCP连接请求的发起者(也就是SYN包)之间出现严重的不平衡时,发起者ISP通常会向接收者ISP支付费用,这在一定程度上促使后者帮助支持那些托管流行内容的ISP。然而,在实践中,这种激励结构通常导致在服务器室内放置一个免费的\emph{pr0n}或\emph{盗版软件}服务器,以降低SYN包计数器的规模。迎合小众受众的博客没有竞争的余地,通常被冷落。然而,请注意,在那个时候,创作者-出版商通常仍然拥有他们的内容。

\subsection{Web 2.0 \statusgreen}\label{sec:web_2}

向\gloss{Web 2.0}的转变很大程度上改变了这一点。从运行在自己的服务器上使用提姆·伯纳斯李的优雅简单和容易的语法编写个人主页向使用cgi-gateways,perl和php编写服务器端脚本转移,使任何人都可以使用简单的工具编写和运行他们自己的网站。这使网络走上了一条令人望而却步的、日益复杂的脚本语言和数据库堆叠的道路。突然间,万维网不再是一个对初学者友好的地方,同时,新技术开始使创建网络应用程序成为可能,这些程序可以提供简单的用户界面,使不熟练的出版商能够简单地将他们的数据发布到服务器上,并使他们摆脱了向最终用户实际交付数据的责任。这样一来,Web2.0就诞生了。

像MySpace和Geocities这样的网站抓住了网络最初的创造者精神,现在占据了主导地位。这些网站为用户提供了一块属于他们自己的互联网空间,在这里有许多滚动的文本框,闪烁的粉色闪光的漫画字体横幅,以及所有脚本小子都能想到的巧妙的跨站攻击。这是一个网中之网,一个可访问的开放环境,用户可以开始越来越多地发布他们自己的内容,而不需要学习HTML,也不需要规则。平台遍地都是,突然间就有了一个适合所有人的地方,一个phpBB论坛,可以满足所有人的兴趣。网络变得生机勃勃,互联网的繁荣使硅谷财富如雨。

当然,这种年轻的天真、令人难以置信的彩虹色游乐场不会持续太久。臭名昭著的不可靠的MySpace成了允许脚本的开放政策的牺牲品。用户的页面变得不可靠,平台也无法使用。当Facebook以一个整洁的界面出现时,MySpace的时代已经结束,人们开始成Swarm结队地迁移。广受欢迎的互联网披上了一层更加自负的色彩,我们鱼贯进入了Facebook的纯白色公司办公室。但是,麻烦即将来临。在“免费”提供这项服务的同时,扎克伯格和其他人也有自己的计划。作为托管我们数据的回报,我们(愚蠢的\cite{carlson2010ims})必须信任他。显然,我们所做的。目前,除了吸引更多的风险投资,积累庞大的用户基础外,表面上还没有商业模式,我们稍后会处理这个问题。但从一开始,广泛的、不可读的条款和条件就把所有内容的权利交给了平台。在Web 1.0时代,你可以很容易地备份你的网站,然后迁移到一个新的主机上,或者干脆自己在家托管它,现在那些有争议的观点有一个新的动词要处理:“deplatformed”。

在基础设施层面,这种集中化开始在难以想象的巨大数据中心中显现出来。杰夫·贝佐斯(Jeff Bezos)通过帮助那些无法解决实施日益复杂和昂贵的基础设施的技术和财务障碍的人,发展了他的图书销售业务,成为了地球上最富有的人。无论如何,这个新的“星座”能够应对那些不规则的流量峰值,这些流量峰值曾在过去严重削弱了广受欢迎的内容。当其他公司紧随其后,很快,大量的网络开始由少数大公司托管。企业收购和源源不断的风投资金使得权力越来越集中。一个被遗忘的开放源码程序员联盟,他们创造了免版税的Apache网络服务器,而谷歌则提供了改变模式的方法来组织和访问指数级扩散的数据,这有助于对微软试图迫使网络进入地狱式的、专有的存在,并永远被囚禁在IE6中的企图进行破坏性打击。当然,谷歌最终接受了“父母的监督”,搁置了“不作恶”的承诺,屈服于自己的狂妄自大,开始吞噬竞争对手。渐渐地,电子邮件变成了Gmail,在线广告变成了AdSense,谷歌渗入了网络日常生活的方方面面。

表面上,一切都很美好。技术乌托邦以一种没有人能够想象的方式将世界超级连接起来。网络不再只是学术界和超级1337的专利,它使任何人都能获得人类知识的总和,现在智能手机变得无处不在,它可以在任何地方被访问。维基百科给了每个人超人的知识,谷歌让我们可以在瞬间找到并访问它,而Facebook让我们可以免费与我们认识的每个人交流。然而,在这一切的背后,隐藏着一个问题。谷歌知道他们在做什么。亚马逊、Facebook和微软也是如此。自1984年以来,一些朋克也是如此。

一旦大型平台拥有了所有用户,就到了给投资者开支票的时候了。制定商业模式的时机已经到来。为了给股东提供价值,内容提供平台将广告收入视为万灵药。谷歌可能真的尝试过,但想不出任何替代方案。现在,网络开始变得复杂,让人分心。广告随处可见,粉红色的闪闪发光的横幅又出现了,这次把你的注意力从你要找的内容上拉开,把你送到下一个用户获取机会上。

似乎这还不够,还有更多的恐怖即将到来。当数据的激增变得难以处理时,互联网失去了它的最后一丝纯真,而算法被提供来“帮助”我们更好地访问我们想要的内容。现在这些平台拥有我们所有的数据,它们能够分析这些数据,得出我们想要看到的东西,似乎比我们了解自己还要了解我们。每个人都会吃到他们最喜欢的零食,以及他们最有可能购买的产品。这些秘密算法和所有相关数据集都有一个陷阱:它们将出售给出价最高的人。财力雄厚的政治组织能够以前所未有的准确性和有效性瞄准摇摆不定的选民。网络空间突然变成了一个非常现实的东西,就像共识和常态一样成为了过去。新闻不仅变成了假的,而且是以个人为目标的操纵,往往在不知不觉中促使你采取违背你最佳利益的行动。

为了节省托管成本,每个人都成为了目标,成为了易于控制的傀儡。一些交易。

与此同时,还有更多可怕的事情在等着我们。事实证明,推动信任互联网最初建设的平等主义理想是最幼稚的。实际上,国防部已经把它给你了,现在想要回来。爱德华·斯诺登(Edward Snowden)带着一大堆谁也想不到的文件走出了美国国家安全局(NSA)。当然,如果你把《谍影重重》当成纪录片的话。结果是协议被破坏了,所有的逮捕令预警早就失效了——世界各国政府一直在对全世界的人口进行监视——不断地存储、处理、编目、索引并提供对一个人在线活动总数的访问。只要触碰XKeyStore按钮,就可以获得所有信息,这是一种视觉上的、不眨眼的视觉神经,它决心“收集所有信息”并“了解所有信息”,不管背景是谁或什么。老大哥看起来很像索伦。
在此之前,世界各地各种权力滥用或狂妄自大的机构和个人采取了许多其他类似的行动,以追踪和封锁被压制的人、政治对手或记者的数据包,成为实行高压政策政权的目标。对隐私的严重侵犯为Tor项目提供了动力。作为回应,美国军方、麻省理工学院和电子前沿基金会之间的这种不同寻常的合作,不仅提供了一种模糊请求来源的方法,而且还以一种受保护的匿名方式提供内容。它在一些利基领域非常成功,家喻户晓,但由于其固有的低效率导致了相对较高的延迟,在其他领域并没有多大用处。

到斯诺登泄密的时候,网络已经无处不在,几乎与人类生活的每个方面都密不可分,但其中绝大部分是由企业运营的。虽然可靠性问题已经成为过去,但这是要付出代价的。上下文敏感的、有针对性的广告模式现在把他们浮士德式的交易延伸到内容生产者身上,他们知道没有其他选择。他们鼓吹"我们将为你提供可扩展的主机,它将应付你的观众抛出的任何流量","但作为回报,你必须让我们控制你的内容:我们将跟踪你的每个观众,并收集(和拥有,*口哨*)尽可能多的他们的个人数据,因为我们能够做到。当然,我们将决定谁能看到它,谁不能看到它,这也是我们的权利。我们会主动进行审查,并在谨慎保护我们业务的情况下与有关部门共享您的数据。”结果,数以百万计的小型内容生产者为一小撮大公司创造了巨大的价值,而得到的回报却微不足道。通常情况下,是免费托管和广告。多么好的交易啊!

暂且不谈Web 2.0数据和我们今天看到的新闻灾难带来的恐惧感,这个体系结构也存在一些技术问题。企业的做法产生了中心主义的准则,因此,现在所有的请求都必须通过某个地方的骨干网,送到一个单一的数据中心,然后传递、处理,最后再返回来。即使只是给隔壁房间的人发个信息。这是一种客户端-服务器架构,它的安全性也很脆弱,而且经常被攻破,以至于它成为了一种新常态,未加密的个人数据甚至是明文密码的泄漏都在网络上传播开来。最后一根钉子钉在棺材是不连贯的标准和接口的蔓延所促进的。如今,越来越复杂的代码实现将web细分为五花八门的微服务。即使是那些财大气长的人也发现越来越难以处理开发费用,而且现在,羽翼未丰的初创企业被迅速致命的、螺旋上升的技术债务淹没在特特性工厂的海洋中是很常见的。现代网络应用程序栈在所有情况下都是一个拼凑起来的Heath-Robinson机器,包括如此多的部件,即使是作为一个超国家的公司,也几乎不可能在维护和开发这些实现时没有无数的错误和经常性的安全缺陷。嗯,除了谷歌和亚马逊。无论如何。现在是重启的时候了。最终,这是描述我们生活的数据。他们已经试过了,但他们没有权力把我们困在这个乱局中。


\subsection{对等网络\statusgreen}\label{sec:peer_to_peer}

随着中心化的Web 2.0风靡全球,\gloss{P2P}(\gloss{peer-to-peer})革命也在加快步伐,悄无声息地并行发展。实际上,P2P流量很快就占据了流经管道的大部分数据包,很快就超过了上面提到的SYN-bait服务器。如果有什么不同的话,那就是毫无疑问的是,最终用户通过合作使用他们迄今尚未充分利用的\emph{上带宽},可以为他们的内容提供同样的可用性和吞吐量,而以前只有在大公司和他们的数据中心的帮助下才能实现,因为他们是互联网主干网最强大的管道。更重要的是,它可以用很低的成本实现。重要的是,用户对他们的数据保留了更多的控制和自由。最终,这种数据分发模式被证明是有弹性的,即使面对强大的、资金充足的实体不遗余力地想关闭它。

然而,即使是\gloss{P2P}文件共享的最先进模式,无跟踪\gloss{BitTorrent} \cite{pouwelse2005bittorrent},也只是文件级共享。这一点也不适合提供人们期望从\gloss{Web 2.0}上的web应用程序中获得的那种交互性、响应性体验。此外,虽然BitTorrent变得非常流行,但它的构想并没有考虑到经济学或博弈论,例如,在世界注意到它的名字将催生的革命之前的时代:也就是说,在人们了解区块链和加密货币的能力和激励之前。\subsection{BitTorrent的经济和它的限制}

bt的天才在于其聪明的资源优化\cite{cohen2003incentives}:如果很多客户想从您下载相同的内容,给他们不同的部分,在第二个阶段,让他们以牙还牙的方式交换彼此之间的部分内容,直到每个人都有所有的内容。这样,无论有多少客户想同时下载内容,用户托管内容(BitTorrent术语为\gloss{seeder})的上游带宽使用基本上总是相同的。这解决了\gloss{HTTP}(支撑\gloss{World Wide Web}的协议)古老的、集中式的主从设计中最棘手的、根深蒂固的问题。

通过使用分层的、分段的哈希算法,可以阻止作弊(也就是给你的对端提供垃圾) 行为,即提供下载的包由一个简短的哈希算法识别,其任何部分都可以被加密证明为包的特定部分,而不需要知道其他部分,并且只需要很小的计算开销。

但是,这种漂亮的简单方法有五个相应的缺点,\cite{locher2006free,piatek2007incentives},这些缺点相互关联。

\begin{itemize}
\item \emph{缺乏经济激励}——
对于下载内容,没有内置的激励机制。特别是,一个人不能将自己的内容播种所提供的上行带宽与下载其他内容所需的下行带宽交换。实际上,通过做种向其他用户提供内容的上行带宽并没有得到奖励。因为尽可能多的上行带宽能够提高某些网络游戏的体验,所以选择放弃游戏是一种理性的选择。再加上懒惰,它就永远不会出现了。

\item \emph{初始延迟}——
通常情况下,下载开始缓慢,有一些延迟。在下载方面比较领先的客户对新来者提供的服务明显多于后者能提供的回报。也就是说,新玩家还没有什么东西可以下载。这样做的结果是,BitTorrent下载从涓涓细流开始,然后变成了全面的比特流。这种特性严重限制了对交互式应用程序的使用,这些应用程序需要快速响应和高带宽。尽管这对于许多游戏来说都是一个绝妙的解决方案。
 
\item \emph{缺乏细粒度的内容寻址}——数据块(\glossplural{chunk})只能作为它们所在的大文件的一部分共享。它们可以被定位为目标,使文件的其余部分可以优化访问。但是下载的对等点只能通过查询\gloss{DHT} (\gloss{distributed hash table})来找到所需的\emph{文件}。在块级别上寻找对等点是不可能的,因为可用内容的广告只发生在文件级别上。这将导致效率低下,因为相同的数据块经常会出现在多个文件中。因此,虽然理论上所有拥有块的节点都可以提供它,但没有办法找到这些节点,因为只有它的封包文件有名称(或者更确切地说,是宣布的散列),可以查找。

\item \emph{没有继续分享的动力}——
一旦节点实现了它们的目标(即从对端那里检索所有需要的文件),它们的共享工作(存储和带宽)就不会得到奖励\item \emph{没有隐私或歧义}——
节点准确地宣传它们正在播种的内容。攻击者很容易发现承载他们想要删除的内容的对等点的IP地址,然后作为DDOS攻击对手的一个简单步骤,他们或企业和国家向ISP请求连接的物理位置。这导致了一个灰色市场,VPN提供商帮助用户规避这一问题。尽管这些服务提供了隐私保证,但由于它们的系统通常是封闭的,因此通常不可能验证它们。 
\end{itemize}

可以说,BitTorrent虽然非常流行,非常有用,但同时也是原始的,是天才的第一步。它是在没有适当的计算和索引的情况下,通过共享我们的上行带宽、硬盘空间和少量的计算能力,我们能走多远。然而,惊喜!-如果我们再加入一些新兴的技术,当然最重要的是\gloss{blockchain},我们就得到了一个真正配得上\gloss{Web 3.0}这个名字的东西:一个去中心化的、抵制审查的共享设备,也可以用于集体创建内容,同时保持对内容的完全控制。更重要的是,这一成本几乎完全由使用和共享由您已经拥有的惊人强大的、未充分利用的超级计算机(根据过去的标准:-)提供的资源来支付。

\subsection{面向Web 3.0 \statusgreen}\label{sec:towards-web3}

% 0/ intro talk about the limitations and problems with web2 app architecture 
% 1/ why the game has changed in a post-satoshi world
% As the blockchain has brought us the ability to 
% 2/ why Swarm represents a further iteration on this change and makes the whole thing usable, how it overcomes limitations of the blockchain, emphasis the VC problem, talk about making the web fun again
% 3/ some short exploration of current attempts to provide this and their potential limitations, but keep this short, unemotional and unbiased
% 4/ drum up to grand ending of how Swarm will provide trustless computing save the world etc. etc.

\begin{centerverbatim}
The Times 03/
Jan/2009 Chancel
lor on brink of 
second bailout f
or banks    
\end{centerverbatim}

2009年1月3日星期六6:15,世界永远地改变了。一个神秘的密码朋克创造了环绕整个世界的第一个链条,精灵就从瓶子里出来了。这第一步将引发一系列反应,导致前所未有的巨额资金从法币和实物商品的传统储物库流向一种全新的存储和传递价值的工具:加密货币。“中松聪”成功地做到了其他人无法做到的事情,事实上,他小规模地将银行的中介化,去中心化了不值得信任的价值转移,从那一刻起,我们实际上又回到了金本位:现在每个人都可以拥有央行的货币。没有人能从你的口袋里增加或膨胀的钱。更重要的是:现在每个人都可以自己印钞票,并配有自己的中央银行和电子传输系统。我们仍不清楚这将在多大程度上改变我们的经济。

这第一步是巨大的,也是一个里程碑式的转折点。现在我们已经将身份验证和价值转移作为系统的核心。但尽管它在概念上很出色,但它在实用性上存在一些不那么小的问题。它可以传输数字价值,人们甚至可以给硬币“着色”,或者发送像上面这个标志着第一个区块的决定性日期的短信息。但就是这样。至于规模……每个事务必须存储在每个节点上。分片不是内置的。更糟糕的是,数字货币的保护要求每个节点的处理方式始终与其他节点完全相同。这与并行计算集Swarm相反,速度要慢数百万倍。

当Vitalik构思以太坊时,他接受了其中的一些限制,但该系统的实用性有了巨大的飞跃。他增加了通过\gloss{Ethereum Virtual Machine}进行图灵完整计算的功能,这使得大量的应用程序可以在这种不可信的设置中运行。这一概念是一次令人眼花缭乱的范式转变,也是比特币的持续进化,比特币本身是基于一个微型虚拟机,每一笔交易实际上都是一个小程序——很多人都不知道。但以太坊一路走下去,再次改变了一切。这种可能性是无数且诱人的,\gloss{Web 3.0}由此诞生。

然而,在完全脱离Web 2.0世界时,仍然有一个问题需要克服:在区块链上存储数据的成本非常高,除了少量的数据之外。比特币和以太坊都采用了BitTorrent的布局并使用它运行,以交易的能力补充了体系结构,但将任何非系统数据的存储考虑留到以后。事实上,比特币在区块分布之下增加了第二条安全程度低得多的电路:候选交易以二等公民的身份在没有任何宣传的情况下进行,实际上没有任何协议。以太坊更进一步,将头从区块中分离出来,创建了第三层,根据需要临时运送实际的区块数据。因为这两类数据对系统的运行都是必不可少的,所以可以称之为关键的设计缺陷。比特币的创造者可能没有预见到,挖矿已经成为高度专业化精英的专属领域。任何交易者都被认为基本上能够挖掘他们自己的交易。以太坊面临着数据可用性的更大挑战,大概是因为这个问题显然可以稍后单独解决,所以暂时忽略它。 

另一方面,\gloss{ZeroNet}和\cite{zeronet}已经成功地将\gloss{BitTorrent}的直接数据传播方法应用于网络内容传播。然而,由于BitTorrent的上述问题,ZeroNet最终无法支持web服务用户所期望的响应性。

为了尝试启用响应,分布式网络应用程序(\glossplural{distributed web application}或\glossplural{dapp}), \gloss{IPFS} (\gloss{InterPlanetary File System}) \cite{ipfs2014}介绍了他们自己对BitTorrent的主要改进。一个突出的特点是高度web兼容,基于url的检索方案。此外,可用数据的目录、索引(像BitTorrent组织为\gloss{DHT})也得到了极大的改进,使搜索任何文件的一小部分成为可能,称为\gloss{chunk}。

有许多其他的努力来填补这一缺口,并提供一个有价值的Web 3.0星座的代理服务器和服务,将由一个Web 2.0开发人员,提供一个解放之路从现有依赖集中式架构,实现数据收割者。这些都不是无足轻重的角色,即使是今天最简单的web应用程序也包含了大量的概念和范例,必须重新映射到web 3.0的\gloss{trustless}设置中。在许多方面,这个问题可能比在区块链中实现不可信的计算更加微妙。Swarm用一系列精心设计的数据结构来回应这一问题,这些结构使应用程序开发人员能够在Web 3.0的新环境中重新创建我们在Web 2.0中已经习惯的概念。Swarm成功地重新构想了当前的网络产品,并在坚实的加密经济基础上重新实施。

想象一个滑动的比例,从左边开始:较大的文件大小,较低的检索频率和更完整的\gloss{API};右边:小数据包、高检索频率和微妙的API。在这个范围内,文件存储和检索系统(如posix文件系统、S3、Storj和BitTorrent)位于左边。键值存储如LevelDB和数据库如MongoDB或Postgres位于右侧。为了构建一个有用的应用程序,需要在不同的范围内使用不同的模式,而且必须有能力在必要时合并数据,并确保只有授权方可以访问受保护的数据。在一个中间派模型中,一开始处理这些问题很容易,随着增长会变得更加困难,但规模的每个范围都有一个或另一个专门软件的解决方案。然而,在分散的模型中,所有的赌注都是不可能的。授权必须用密码学处理,数据的组合也因此受到限制。因此,在今天初生的、不断发展的Web 3.0堆栈中,许多解决方案只处理这一需求范围的一部分。在本书中,您将了解Swarm如何跨越整个领域,以及如何为Web 3.0开发人员的新卫士提供高级工具。希望从基础架构的角度来看,在Web 3.0上工作就像Web 1.0的宁静日子一样,同时提供前所未有的代理、可用性、安全性和隐私级别。

为了响应文件共享的根级对隐私的需求——因为它在以太坊中非常有效——Swarm在同样基本和绝对的级别强制匿名。Web 2.0的教训告诉我们,信任应该谨慎地给予,并且只给予那些值得信任的人,并且会以尊重的态度对待信任。数据是有毒的abbeb,我们必须小心对待它,以便对我们自己和我们负责的人负责。稍后我们将解释Swarm如何提供完整和基本的用户隐私。

当然,要完全过渡到Web 3.0去中心化的世界,我们还需要处理激励和信任的维度,传统上通过将责任移交给(通常不值得信任的)中央把关人来“解决”这些问题。正如我们所注意到的,这也是BitTorrent努力解决的一个问题,它用过多的种子比率和私人(即中央)跟踪来回应。

在ZeroNet或MaidSafe等各种项目中,缺乏可靠的托管和存储内容的动机的问题很明显。分布式文档存储的激励仍然是一个比较新的研究领域,特别是在区块链技术的背景下。洋葱网络已经看到了\cite{jansen2014onions,ghoshetal2014tor}的建议,但这些方案主要是学术的,它们没有内置到底层系统的核心。比特币已经被重新用于驱动其他系统,如Permacoin \cite{miller2014permacoin},一些已经创建了自己的区块链,如Sia \cite{vorick2014sia}或Filecoin \cite{filecoin2014}的\gloss{IPFS}。BitTorrent目前正在用他们自己的代币\cite{tron2018,bittorrent2019}测试区块链激励机制。然而,即使将所有这些方法结合在一起,仍有许多障碍需要克服,才能为Web 3.0 dapp开发人员提供特定的需求。

稍后我们将看到Swarm如何提供一套完整的激励措施,以及其他制衡措施,以确保节点的工作有利于整个……Swarm。这包括选择大量的磁盘空间出租给那些愿意付钱——不管其内容的流行,同时确保还有一个方式来部署您的交互式动态内容存储在云中,我们称之为上传即消失(\gloss{upload and disappear})。

任何对\gloss{peer-to-peer}内容分配的激励制度的目标都是鼓励合作行为和阻止搭便车(\gloss{freeriding}):有限资源的无补偿耗竭。这里概述的激励策略(\gloss{incentive strategy})希望满足以下限制条件:

\begin{itemize}
    \item 无论其他节点是否遵循它,这都符合节点自身的利益。
    \item 花费其他节点的资源一定很昂贵。
    \item 它不会强加不合理的开销。
    \item 它可以很好地处理“幼稚的”节点。
    \item 它奖励那些表现良好的人,包括那些遵循这一策略的人。
\end{itemize}

在Swarm环境下,存储和带宽是两个最重要的有限资源,这在我们的激励方案中得到了体现。带宽使用的激励措施旨在实现快速和可靠的数据供应,而存储激励措施旨在确保长期数据保存。通过这种方式,我们确保web应用程序开发的所有需求都得到满足,并且激励机制是一致的,这样每个单独的节点的行动不仅有利于它自己,而且有利于整个网络。 

\section{公平的数据经济}\label{sec:fair-data}
\green{}

在\gloss{Web 3.0}时代,互联网不再只是极客们玩的小众游戏,而是成为价值创造的基本渠道和整体经济活动的巨大份额。然而,目前的数据经济远非公平,利益的分配是在那些控制数据的人的控制下进行的——大多数公司将数据单独保存在数据井(\glossplural{data silo})中。为了实现\gloss{fair data economy}的目标,必须解决许多社会、法律和技术问题。现在我们将描述一些目前存在的问题,以及Swarm将如何解决这些问题。 

\subsection{数据经济的当前状态\statusgreen} \label{sec:dataeconomy}

数字镜像世界已经存在,虚拟空间包含物理物体的阴影,并由难以想象的大量数据\cite{MirrorWorlds2020Feb}组成。然而,越来越多的数据将继续同步到这些平行世界,这就需要新的基础设施和市场,并创造新的商业机会。目前只有相对粗糙的方法可以衡量整个数据经济的规模,但对于美国来说,有一个数字表明,2019年数据(包括相关软件和知识产权)的金融价值在1.4万亿美元到2万亿美元之间。欧盟委员会预计,欧盟27国2025年的数据经济将达到8290亿欧元(高于2018年的3010亿欧元)。

尽管存在如此巨大的价值,但现有数据经济所产生的财富分配的不对称已被作为一个重大的人道主义问题提出。由于数据的改善,效率和生产率不断提高,由此产生的利润将需要进行分配。如今,收益分配不均:公司的数据集越大,从中可以学到的东西就越多,从而吸引更多用户,从而获得更多数据。目前,这一点在\gloss{FAANG}等占主导地位的大型科技公司身上表现得最为明显,但据预测,这在非技术领域,甚至在民族国家,也将变得越来越重要。因此,企业正竞相在某个特定领域占据主导地位,而拥有这些平台的国家将获得优势。正如联合国贸易和发展会议(United Nations Conference on Trade and Development \cite{TheWinner2020Feb})所警告的那样,由于非洲和拉丁美洲的此类机构如此之少,它们有可能成为原始数据的出口国,然后付钱给其他国家进口所提供的情报。另一个问题是,如果一家大公司垄断了一个特定的数据市场,它也可能成为数据的唯一购买者——保持对定价的完全控制,并提供一种可能性,即提供数据的“工资”可能被人为压低。在许多方面,我们已经看到了这方面的证据。 

% move this?
% As a solution, citizens could organise into "data co-operatives", who would then act as trade unions do in conventional economy. 

数据流正越来越多地受到政府的屏蔽和过滤,这是基于对公民、主权和国家经济的保护的常见推理。几位安全专家泄露的信息表明,政府要适当地考虑国家安全,数据应该保存在家门口,而不是留在其他国家。GDPR就是已经建立的“数字边界”的一个例子——只有在适当的保护措施到位的情况下,数据才能离开欧盟。其他国家,如印度、俄罗斯和中国,在数据上也有自己的地理限制。欧盟委员会已承诺密切监测这些国家的政策,并通过世界贸易组织(\cite{EUWhitePaperAI2020Feb})的行动,解决贸易谈判中对数据流动的任何限制或限制。

尽管人们对数据的潮起潮落越来越感兴趣,但大型科技公司仍牢牢掌握着大部分数据,而魔鬼就在细节中。Swarm的隐私第一模式要求个人数据不泄露给任何第三方,一切都是端到端加密的,终结了服务提供商聚集和利用庞大数据集的能力。这样做的结果是,数据的控制不是集中在服务提供商,而是分散的,并与相关的个人有关。其结果是,权力也是如此。预计负面新闻。

\subsection{数据主权的当前状态和问题\statusgreen }\label{sec:data-sovereignty}

上述浮士德式交易的结果是,美联储现行的模式在许多方面都存在缺陷。作为基础设施供应中的规模经济以及社交媒体中的网络效应在很大程度上不可预见的后果,平台成为了巨大的数据竖井,在这里,大量用户数据通过并保存在属于单一组织的服务器上。集中式数据模型的“副作用”让大型私营企业有机会收集、聚合和分析用户数据,将他们的数据吸管正好放在中心瓶颈:云服务器。这正是David Chaum在1984年所预言的,并由此开启了《Swarm》灵感的Cypherpunk运动。

以计算机为媒介的互动取代人类为媒介的互动的持续趋势,再加上社交媒体和智能手机的兴起,使得提供数据流的公司能够轻易获取越来越多关于我们个人和社会生活的信息。这些都揭示了有利可图的数据市场,用户统计数据与潜在行为联系在一起,从而更好地了解你,而不是你自己。这是市场营销人员的宝库。

与此同时,数据公司的商业模式也在朝着通过销售数据而不是最初提供的服务获利的方向发展。现在,他们的主要收入来源是向广告商、营销人员和其他寻求“推动”公众的人出售用户信息。通过在同一个平台上向用户展示这样的广告,衡量他们的反应,从而形成一个反馈循环,圈就被关闭了。一个全新的行业已经从这一信息洪流中成长起来,结果,复杂的系统出现了,这些系统可以预测、引导和影响用户如何分配他们的注意力和金钱,公开且知情地利用人类在对刺激做出反应时的弱点,经常诉诸于高度发达和精心策划的心理操纵。事实是,毫无疑问,这是一种以商业为名的大规模操纵,在这种情况下,即使是最清醒的人也无法真正行使他们的选择自由,并保持他们在内容消费或购买习惯方面的内在偏好自主权。

企业收入来自于向微目标用户展示广告的需求,这一事实也反映在服务质量上。内容用户的需求——过去是并且应该继续是“最终”用户——变得次要于“真正”用户的需求:广告商,这往往导致用户体验和服务质量越来越差。对于社交平台来说,这尤其令人痛苦,因为网络效应导致的惯性基本上构成了用户锁定。必须纠正这些错位的激励机制。换句话说,就是向用户提供相同的服务,但不像集中式数据模型那样产生令人遗憾的激励。

对个人数据缺乏控制会对用户的经济潜力造成严重后果。有些人有点歇斯底里地把这种情况称为\gloss{data slavery}。但从技术上讲,它们是正确的:我们的数字双胞胎被企业俘虏,被它们利用得很好,而我们没有任何代理,相反,操纵我们脱离它,让我们变得不那么灵通和自由。

那么,当前的系统,保持数据在断开的数据集有各种缺点: 

\begin{itemize}
    \item 集中化的实体增加了不平等,因为他们的系统从价值的实际创造者那里吸走了不成比例的利润。
    \item \emph{缺乏容错能力} -从技术基础设施,尤其是安全方面来说,它们是单点故障。
    \item 决策权的集中更容易成为社会工程、政治压力和制度化腐败的目标。
    \item \emph{单一的攻击目标}—在同一个安全系统下集中大量的数据会引起攻击,因为它增加了黑客的潜在回报。 
    \item \emph{缺乏服务连续性保证} -服务连续性掌握在组织手中,仅靠声誉来激励。这就引入了由于破产、监管或法律行动而无意终止服务的风险。
    \item 数据访问的集中控制会导致言论自由的减少,在大多数情况下最终会导致言论自由的减少。
    \item 通过中央所有的基础设施流动的数据为流量分析和其他监控方法提供了完美的访问。
    \item \emph{操纵} -显示层的垄断性使得数据收集者可以通过选择哪些数据以什么顺序和什么时候显示来操纵意见,从而质疑个人决策的主权。
\end{itemize}


\subsection{走向自主数据\statusgreen} \label{sec:selfsovereigndata}

我们相信去中心化是一个主要的游戏规则改变者,它本身就解决了上面列出的许多问题。

我们认为区块链技术是实现真正自主互联网的密码朋克理想的最后一块缺失的拼图。正如Eric Hughes在1993年的\emph{Cypherpunk宣言} \cite{hughes1993}中所说,“我们必须团结起来,创建允许匿名交易的系统。”本书的目标之一是演示如何将分散的共识和点对点网络技术结合起来,形成一个坚如磐石的底层基础设施。这个基础不仅具有弹性、容错和可扩展性;而且是平等的,经济上可持续的,有一个精心设计的激励系统。由于参与者的进入门槛较低,这些激励措施的适应性确保了价格自动收敛到边际成本。除此之外,还有Swarm在隐私和安全领域的强大价值主张。

Swarm是一个去中心化、激励和安全的\gloss{Web 3.0}堆栈。特别是,该平台为参与者提供了数据存储、传输、访问和身份验证的解决方案。这些数据服务在经济互动中越来越重要。通过向所有人提供这些服务,并提供强有力的隐私保障,没有国界或外部限制,Swarm培养了全球自愿精神,代表了\emph{一个自主的数字社会的基础设施}。

\subsection{人工智能和自主数据\statusgreen} \label{sec:AIdata}

人工智能(AI)有望给我们的社会带来重大变化。一方面,它被设想为无数的商业机会,而另一方面,它被期望取代许多职业和工作,而不仅仅是增加他们\cite{Lee2018Sep}。

当今流行的人工智能(机器学习)所需要的三个“要素”是:计算能力、模型和数据。今天,计算能力是很容易得到的,专门的硬件正在开发,以进一步促进处理。一场大规模的人工智能人才猎头活动已经进行了十多年,企业成功地垄断了拥有提供模型和分析所需专业人才的员工。然而,今天的人工智能和深度学习的肮脏秘密是,算法,即“智能数学”已经商品化。它是开源的,而不是谷歌或Palantir用来赚钱的。释放他们超能力的真正“魔术”是获得尽可能多的数据集。

恰巧的组织系统地收集数据竖井,经常通过向用户提供应用程序等一些实用的搜索和社交媒体,然后储存数据供以后使用完全不同于“用户”没有他们想象的表达同意和不知识。这种对数据的垄断,让跨国公司获得了前所未有的利润,而与这些公司出售数据的人分享金融收益的动议却寥寥无几。但更糟糕的是,他们储存的数据无法实现其潜在的变革价值,不仅对个人,对整个社会都是如此。

也许这不是巧合,因此,主要的数据和人工智能“超级大国”正在以美国和中国的政府以及总部设在那里的公司的形式出现。在全世界人民的注视下,一场人工智能军备竞赛正在展开,几乎所有其他国家都被甩在了后面,成为“数据殖民地”\cite{HarariDavos2020Mar}。有人警告说,按照目前的情况,中国和美国将不可避免地积累不可逾越的优势,成为人工智能超级大国\cite{Lee2018Sep}。

其实不必如此。事实上,它可能不会,因为现状对数十亿人来说是一个糟糕的交易。去中心化技术和密码学是一种允许数据隐私的方式,同时使公平的数据经济逐渐出现,它将呈现当前集中化数据经济的所有优点,但没有有害的缺点。这是全球许多消费者和科技组织都在努力争取的变化,以支持对大数据巨头的反击,越来越多的用户开始意识到,他们是被欺骗而泄露数据的。蜂Swarm将提供基础设施来促进这一解放。

自我主权存储可能是个人能够重新控制他们的数据和隐私的唯一方式,作为重获自主权的第一步,走出过滤泡沫,重新与自己的文化联系起来。Swarm是解决当今互联网和数据分布和存储问题的核心解决方案。它从头到尾都是为隐私而建的,数据加密复杂,通信完全安全、防泄漏。此外,它可以在个人的条件下与第三方共享选定的数据。支付和激励是Swarm的组成部分,以经济补偿来换取数据的细粒度共享是一个核心问题。

因为正如休斯所写,“开放社会中的隐私需要匿名交易系统. ...。匿名交易系统不是秘密交易系统。一个匿名系统使个人能够在需要时和仅在需要时透露自己的身份;这就是隐私的本质。”

使用Swarm可以利用更完整的数据集,创造更好的服务,同时还可以选择以可自我验证的匿名方式为全球利益做出贡献。世界上最好的。

这种新的更广泛的可用性的数据,如年轻学术和学生创业与颠覆性的创意在AI和大数据领域工作,将极大地促进整个领域的演变作了不少贡献科学,医疗、消除贫困、环境保护、灾害预防、等等;但是,尽管强盗大亨和流氓国家取得了引人注目的成功,但它目前处于僵局。有了Swarm提供的设施,公司和服务提供商将有一系列不同但同样强大的新选择。随着数据的广泛去中心化,我们可以共同拥有极其庞大和有价值的数据集,这些数据集是构建最先进的人工智能模型所需的。这种数据的可移植性已经在传统技术中得到了暗示,它将使竞争成为可能,并像以前一样为个人提供个性化服务。但竞争环境对所有人来说都是公平的,这将推动值得在2020年进行的创新。 



\subsection{集体信息\statusgreen}\label{sec:collective_information}

\glossupper{collective information}从互联网首次出现时就开始积累,但这个概念最近才在\emph{开源}、\emph{合理的数据}或\emph{信息共享}等各种标题下得到认可和讨论。

集体信息,由维基百科(本身就是“集体信息”的一个例子)定义为:
\begin{displayquote}
“一组实体,它们至少有一个共同的问题或兴趣,或为实现共同的目标而一起工作。” 
\end{displayquote}
尽管地理位置、政治信仰、社会地位、财富、甚至普遍自由和其他人口特征存在差异,但互联网允许集体以以前无法想象的规模形成。这些集体通过在公共论坛、评论、投票、存储库、文章和投票上的联合互动产生的数据是集体信息的一种形式——从这些互动的痕迹中产生的元数据也是如此。由于今天大多数的这些交互都是由运行集中服务器的营利性实体推动的,这些集体信息最终被存储在商业实体拥有的数据竖井中,其中大部分集中在几家大型技术公司的手中。虽然实际的工作结果往往是公开的,但作为这些提供者的提供,元数据通常是贡献者互动的更有价值、更强大和更危险的代表,通常是秘密持有和货币化的。

在一个数字化的社会中,这些“平台经济”已经变得至关重要,而且只会变得越来越重要。然而,我们看到,商业玩家通过其用户获取的信息正越来越多地被用于损害用户的最大利益。委婉地说,这让人怀疑,这些公司是否有能力承担与保留我们的\gloss{collective information}权力相伴随的道德责任。

尽管许多国家试图不受限制地获取大量个人数据,但有些国家要求获得像魔术钥匙一样的后门访问权,但也有例外。由于人工智能有可能被滥用和在伦理上存在问题的使用,许多国家已经启动了人工智能使用的“伦理”举措、法规和认证,例如德国数据伦理委员会或丹麦的数据伦理印章。

然而,即使可以使企业的行为更值得信赖(鉴于它们的重大责任,这是适当的),\glossplural{data silo}的存在本身就扼杀了创新。客户端-服务器架构本身的基本形状导致了这个问题,因为它使集中式数据存储(在他们的“农场”的“服务器”上)成为默认(参见\ref{sec:web_1}和\ref{sec:web_2})。有效的点对点网络(如Swarm (\ref{sec:peer_to_peer}))现在可以改变这种架构的拓扑结构,从而实现集体信息的集体所有权。 


\section{愿景\statusorange}\label{sec:vision}

\wip{Gregor: still working on text}

\begin{displayquote}
蜂Swarm是一个自治社会的基础设施。 
\end{displayquote}


\subsection{值\statusorange}\label{sec:values}

Self-sovereignty意味着自由。如果我们将其分解,这意味着以下元值:

\begin{itemize}
\item \emph{包容性} -公众和未经允许的参与。  
\item \emph{完整性} -隐私,可证明的来源。 
\item 节点和网络的兴趣对齐。
\item 内容和价值中立。  
\end{itemize}

这些元价值可以被认为是有助于增强个人和集体获得自我主权的系统品质。

包容性意味着我们渴望将弱势Swarm体纳入数据经济;以及降低定义复杂数据流和构建分散应用程序的准入门槛。Swarm是一个开放参与的网络:为发布、共享和投资你的数据提供服务和未经许可的访问。

用户可以自由地将他们的意图表达为“行动”,并拥有完全的权力来决定他们是否想要保持匿名或分享他们的互动和偏好。在线角色的完整性是必需的。

经济激励有助于确保参与者的行为与网络的预期紧急行为相一致(参见\ref{sec:incentivisation})。

公正性保证了内容的中立性,防止了守门行为。它还重申,其他三种价值观不仅是必要的,而且是充分的:它排除了将任何特定Swarm体视为特权或表示偏爱任何特定来源的特定内容或数据的价值观。 

\subsection{设计原则\statusorange}\label{sec:design-principles}
 

信息社会和数据经济带来的时代,在线交易和大数据是必不可少的日常生活,因此支持技术是关键的基础设施。因此,这一基础层基础设施必须是\emph{未来的证明},即为连续性提供强有力的保证。 

通过以下\emph{系统属性}表示的一般要求可实现连续性:

\begin{itemize}
\item \emph{稳定的} -规范和软件实现对参与或政治(政治压力,审查)的变化是稳定和弹性的。
\item \emph{可伸缩的} -与一开始相比,该解决方案能够容纳更多数量级的用户和数据,在大规模采用期间不会导致性能或可靠性的令人望而却步的降低。  
\item \emph{安全}——该解决方案能抵抗蓄意攻击,不受社会压力和政治影响,也能容忍其技术依赖中的错误(例如区块链,编程语言)。 
\item \emph{自我维持的} -该解决方案作为一个自主系统自行运行,不依赖于人或组织协调集体行动或任何法律实体的业务,也不依赖专有技术、硬件或网络基础设施。 
\end{itemize}




\subsection{目标\statusyellow}\label{sec:objectives}

%\subsubsection{Scope}

当我们谈论“数据流”时,其中一个核心方面是信息如何在不同的模式中具有可证明的完整性,参见表\ref{tab:scope}。这与以太坊\gloss{world computer}的原始愿景相对应,构成了未来数据场景的无信任(即完全可信)结构:一个支持数据存储、传输和处理的全球基础设施。

\begin{table}[htb]
\centering
\begin{tabular}{c|c|c}
dimension & model & project area\\\hline
%
time & memory & storage \\
space & messaging & communication \\
symbolic & manipulation & processing \\
\end{tabular}
\caption{Swarm's scope and data integrity aspects across 3 dimensions.}
\label{tab:scope}
\end{table}

 
以太坊区块链作为世界计算机的CPU, Swarm被认为是它的“硬盘”。当然,这个模型掩盖了Swarm的复杂性,它的能力远不止简单的存储,正如我们将讨论的那样。

Swarm项目旨在实现这一愿景,并建立世界计算机的存储和通信。 

\subsection{影响地区\statusorange}

在接下来的内容中,我们试图确定产品的特性区域,以最好地表达或促进上面讨论的价值。

在无许可参与方面的包容性最好由去中心化的点对点网络来保证。
允许节点提供服务并从中获得报酬将为生态系统提供零现金入口:没有货币的新用户可以为其他节点服务,直到他们积累了足够的货币来使用自己的服务。一个不需要看门人就能提供分布式存储的去中心化网络也是包容和公正的,因为它允许那些冒着被专制当局取缔平台风险的内容创造者在不受审查的情况下发表自己的言论自由。

经济激励的系统内置效果最好的协议如果它追踪行动,造成成本的点对点交互:带宽共享体现在消息传送就是这样的一个行动,立即会计是可能的节点收到一个消息,是有价值的。另一方面,诸如承诺保存数据这样的承诺只有在经过核实后才能得到回报。为了避免\gloss{tragedy of commons}问题,应该通过惩罚措施的威胁来加强个人问责,即通过允许参股保险公司,来防范此类承诺。

完整性是由易于证明的真实性,同时保持匿名。
可证明的包含和惟一性是允许不可信数据转换的基础。

% \subsection{Requirements \statusred}\label{sec:requirements}

\subsection{未来} \label{sec:future}

未来是未知的,人类面临着许多挑战。在当今的数字社会中,可以肯定的是,国家和个人要想拥有主权并掌握自己的命运,就必须保留对其数据和通信的访问和控制。

Swarm的愿景和目标源于去中心化的技术社区及其价值观,因为它最初被设计为三巨头中的文件存储组件,这三巨头将形成世界计算机:以太坊、Whisper和Swarm。

它提供了运行在用户设备上的dapps所需的响应能力,以及使用任何类型的存储基础设施(从智能手机到高可用性集Swarm)的良好激励存储。通过精心设计的带宽和存储激励措施,连续性将得到保证。

内容创造者将从他们提供的内容中获得公平的补偿,而内容消费者将为其付费。通过去除目前受益于网络效应的中间商,这些网络效应的好处将会扩散到整个网络。

但它将远不止于此。每个人,每个设备都会留下数据痕迹。这些数据被收集并存储在竖井中,其潜力仅被部分利用,这有利于大型企业。

蜂Swarm将是数字镜像世界的首选场所。个人、社会和国家将有一个云存储解决方案,它将不依赖于任何一个大型提供商。 

% This is especially important for countries currently lagging behind, such as Africa and Latin America. 

个人将能够完全控制自己的数据。他们将不再需要成为数据奴隶,用自己的数据换取服务。不仅如此,它们还能够组织成数据集体或数据合作社——共享某些类型的数据,以达到共同的目标。

各国将建立自主的集Swarm云(Swarm cloud)作为数据空间,以迎合工业、卫生、移动和其他部门新兴的人工智能产业。云将在对等体之间,尽管可能在一个专属区域内,第三方将不能干涉数据和通信的流动——监控、审查或操纵它。然而,任何有适当许可的一方都可以访问它,从而有望为人工智能和基于它的服务创造公平的竞争环境。

矛盾的是,Swarm可以成为存储数据的“中心”场所,使数据的可访问性、控制、价值公平分配和利用数据造福个人和社会。

在未来的社会中,Swarm将变得无处不在,在Fair数据经济中,将个人和设备的数据透明、安全地提供给数据消费者。


用于获取默认内容寻址块地址的散列方法称为\gloss{binary Merkle tree hash},或简称为\gloss{BMT hash}。 

\subsubsection{计算BMT哈希值}

二叉树的基段是块内容数据的子序列。 
段的大小是32字节,这是用于构造树的\emph{基地哈希}的摘要大小。 
给定用于表示文件的Swarm哈希树(参见afaed),假设中间块包括了对其他块的引用。 

获取序列的BMT哈希包括以下步骤:

\begin{enumerate}
\item \emph{填充} -如果内容小于最大块大小(4096字节,\ref{sec:content-addressed-chunks}),则用0填充直到块大小。注意,这个零填充仅用于哈希,不会影响块数据大小。
\item \emph{块数据层} -计算填充块中\emph{双段}的基哈希值,即数据的段大小($2 * 32$)单元,并将结果连接起来。
\item \emph{构建树} -对结果重复上一步,直到结果只是一个部分。
\item \emph{计算跨度} -计算数据的跨度,即包含在由未填充数据表示为64位小端整数值(参见afaed)的块下的数据的大小。            
\item \emph{完整性保护} -将span添加到二叉树的根哈希中,并计算数据的\emph{基地哈希}。
\end{enumerate}

\begin{definition}[BMT hash]\label{def:bmt-hash}
\begin{lstlisting}[language=buzz1]
// /bmt

define function hash payload 
    with span
as
    @padded = @payload as [:chunk size]byte    // use zero padding 
    // for BMT hashing only
    hash @span and root of @padded over chunk size 
    
define function root of @section []byte
    over @len uint
as
    return hash @section        // data level
        if @len == 2 * segment size
    @len /= 2                  // recursive call
    @children = @section each @len go self over @len
    wait for @children 
        join hash
    

\end{lstlisting}
\end{definition}

\subsubsection{包含证明}

将片段与这些块中打包的哈希对齐,可以将包含证明的概念扩展到文件中。
BMT哈希支持紧凑的第三方可验证的段包含证明。

\red{指定}
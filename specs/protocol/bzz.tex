

bzz握手协议是在两个对等体建立连接之后以及在任何其他协议建立之前运行的协议。它传递有关对等体地址、网络ID和轻节点能力的信息。

握手协议只定义了一个流和三个消息:

\begin{definition}[Bzz handshake protocol  messages]\label{def:bzz-messages}

\begin{lstlisting}
// ID: /swarm/handshake/1.0.0/handshake

syntax = "proto3";

package handshake;

message Syn {
    bytes ObservedUnderlay = 1;
}

message Ack {
    BzzAddress Address = 1;
    uint64 NetworkID = 2;
    bool Light = 3;
}

message SynAck {
    Syn Syn = 1;
    Ack Ack = 2;
}

message BzzAddress {
    bytes Underlay = 1;
    bytes Signature = 2;
    bytes Overlay = 3;
}
\end{lstlisting}
\end{definition}

这个消息序列是受TCP三次握手的启发,以确保消息的可传递性。

在连接时,请求的对等体构造一个新的握手流,并发送一个\lstinline{Syn}消息,该消息带有远程观察到的与之进行握手的对等体的底层地址。之后,它等待来自响应对等体的\lstinline{SynAck}响应消息。在\lstinline{SynAck}消息中,响应器也发送它自己的\lstinline{Syn}消息,以及确认消息,确认消息应该包括它正确签名的BzzAddress。接收到的观测地址可以用来与本地已知的地址进行比较,以便在确认中发送更好的可广告地址(Underlay)。请求端从响应端接收到\lstinline{SynAck}消息,并确认接收到的\lstinline{Ack}信息是正确的后,向响应端发送一条\lstinline{Ack}消息作为确认。流在收到\lstinline{Ack}消息后被响应的对等体关闭。

如果网络id不匹配,或者没有遵循消息的确切顺序,则必须终止连接。

验证bzz地址,提取Overlay层、底层和签名。
Light是一个布尔字段,指示节点是否作为一个Light(而不是完整)节点运行。

握手后,每个对等体需要记住对方的以下数据:

\begin{itemize}
    \item Overlay地址-用于转发(参见\ref{spec:strategy:forwarding}),
    \item 底层地址——用于拨号,当连接驱动程序需要连接到对等体时传递给底层网络协议(参见\ref{spec:strategy:connection}),
    \item bzz地址签名——hive协议需要将该节点的信息传递给其他对等体(参见\ref{spec:protocol:hive}),
    \item 对端是否是轻节点。
\end{itemize}

% \subsection{Encapsulation of price information \statusred}


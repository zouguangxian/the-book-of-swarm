\gloss{hive protocol}使节点能够交换与它们相关的其他对等点的信息,以便引导它们的连接(请参阅\ref{sec:bootstrapping})。通信的信息是已知远程对等点的覆盖地址和底层地址(参见\ref{spec:format:bzzaddress})。覆盖地址用于选择对等点,以实现所需的网络拓扑所需的连接模式。通过拨号选定的对等体来建立对等体连接需要底层地址。

\subsection{流和消息\statusgreen}


协议指定一个流有两个消息:

\begin{definition}[Hive protocol messages]\label{def:hive-messages}

\begin{lstlisting}
// /swarm/hive/1.0.0/peers
syntax = "proto3";

package hive;

message Peers {
    repeated BzzAddress peers = 1;
}

message BzzAddress {
    bytes Underlay = 1;
    bytes Signature = 2;
    bytes Overlay = 3;
}

\end{lstlisting}
\end{definition}

在连接的生命周期内,节点可以广播新接收到的对等体。这是通过通过\\\lstinline{/swarm/hive/1.0.0/peers}流发送\lstinline{Peers}消息来实现的。

在接收到对等点消息后,节点将在其\gloss{address book}中存储对等点信息,即包含该节点已知的对等点信息的数据结构。地址簿用于根据连接策略(\ref{spec:strategy:connection})向连接管理器推荐对等点,以引导kademlia拓扑(\ref{sec:kademlia-connectivity})。地址簿意味着跨会话持久化。

\subsubsection{发送端}

创建一个具有适当id的流,并通过流发送\lstinline{Peers}消息。这条消息没有回应。发送节点应该等待接收端关闭自己的流,然后再关闭流并继续前进。

\subsubsection{接收端}

When the stream is created, receiving node should wait for a \lstinline{Peers} message. After receiving the message, node should close its side of the stream to let the sender node know that the message was received, and move on with processing. If the new node was not known, it should also be forwarded to all connected peers closer to peer address then the node themselves.
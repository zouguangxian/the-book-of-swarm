
\subsection{衬底网络\statusgreen}

\lstinline{libp2p}网络堆栈为\ref{sec:underlay-transport}中布局的群底层网络提供了所有必需的属性。

\begin{enumerate}
\item 以\emph{多地址}的形式为每个节点提供寻址,这里称为底层地址。每个节点可以有多个底层地址,这取决于配置的传输和网络侦听地址。
\item 拨号是通过\lstinline{libp2p}支持的网络传输提供的。
\item 监听由\lstinline{libp2p}支持的网络传输提供。
\item 在两个对等体之间建立实时连接,并保持打开状态以接受或发送消息。
\item 通过\lstinline{TLS}和\lstinline{libp2p secio}流安全传输提供信道安全。
\item 协议复用是由\lstinline{libp2p mplex}流复用协议提供的。
\item 通过使用\lstinline{libp2p}双向流来验证对等端对发送消息的响应,可以提供交付保证。
\item \lstinline{libp2p}并不强制序列化,因为它提供了字节流,允许每个协议灵活地选择最合适的序列化。推荐的序列化方式是\lstinline{Protobuf},在流中使用\lstinline{varint}分隔的消息。
\end{enumerate}

\subsection{协议和流\statusgreen}

对等点之间的通信是在协议中组织的,在一个唯一名称下的逻辑单元,可以定义一个或多个\emph{流}。\lstinline{libp2p}提供流作为通信的基本通道。流是字节的全双工通道,在两个对等体之间的单个连接上多路复用。

每一个流程定义:

\begin{itemize}
\item 以semver形式遵循语义版本控制的版本
\item 数据序列化的定义
\item 通过全双工流在对等体之间传递的数据序列
\end{itemize}

流由\lstinline{libp2p}区分大小写的协议id标识。以下约定用于构造流标识符:

\begin{lstlisting}
/swarm/ProtocolName/ProtocolVersion/StreamName
\end{lstlisting}

\begin{itemize}
\item 所有流id都以\lstinline{/swarm}为前缀。
\item \lstinline{ProtocolName}是标识协议的任意字符串。
\item \lstinline{ProtocolVersion}是一个semver形式的字符串,用于指定协议实现之间随时间的兼容性。
\item \lstinline{StreamName}是一个任意字符串,用于标识定义为协议一部分的流。
\end{itemize}

\subsection{数据交换序列\statusgreen}

在一个打开的流下,数据传递序列必须是同步的。可以同时打开多个流,这些流在同一连接上进行多路复用,独立和异步地交换数据。流可以使用不同的数据交换序列,如:

\begin{itemize}
\item \emph{单独的消息发送} -在关闭流之前不等待对等体的响应,如果它不需要。
\item \emph{多个消息发送}—在关闭流之前不从对等端读取而发送到对等端的一系列数据。
\item \emph{请求/响应}—在关闭流之前需要对单个请求的单个响应。
\item \emph{多个请求/响应周期}——在每个请求之后关闭流之前都需要一个同步响应。
\item \ref{spec:protocol:hive} -需要在单个流上以精确的顺序(参见\ref{spec:protocol:hive}中的握手协议)使用多种消息类型。
\end{itemize}

流有预定义的序列,为了一个目的尽可能地保持简单。对于复杂的消息交换,应该使用多个流。

对于即时的数据交换或通信,流的生命周期可能很短;如果需要,流的生命周期可能很长。

\subsection{流头}

所有libp2p流的一个Swarm特定要求是在两个对等体之间的每个流初始化时交换Header protobuf消息。此消息封装了在交换任何特定于流的数据或消息之前需要交换的流作用域信息。标头是键值对的序列,其中键是任意字符串,值是不强加任何特定编码的字节数组。每个键都可以对它所关联的数据使用适当的编码。

\begin{definition}[Header message]\label{def:headers-message}

\begin{lstlisting}[language=protobuf3]
syntax = "proto3";

package pb;

message Headers {
    repeated Header headers = 1;
}

message Header {
  string key = 1;
  bytes value = 2;
}
\end{lstlisting}
\end{definition}

在每个流初始化时,创建它的对等端都发送报头消息,不管它是否包含报头值。接收节点必须读取此消息并使用使用相同消息类型的响应头进行响应。这使得报头交换序列完成,任何其他流数据都可以根据协议传输。

标准头键名定义在这里:

\begin{enumerate}
\item tracing-span-context
\end{enumerate}


\subsection{用于跟踪\statusgreen的上下文的封装}

使用流头来交换P2P流作用域的跟踪跨度上下文。头键" trace -span-context"保留用于二进制编码的跟踪span上下文数据。该上下文应该用于跟踪消息。流启动器节点应该向响应节点提供跟踪跨上下文。该上下文是可选的,所有节点的功能必须与其他节点提供的span上下文相同,无论该节点是否配置了跟踪。

\chapter{绪论\statusgreen}
\green{}
\section*{目标受众\statusgreen}
本书的主要目的是获取Swarm项目第一阶段的成果,并为参与到即将到来的阶段的Swarm项目的团队和个人提供一个纲要。

这本书是为有技术倾向的读者准备的,他们有兴趣在他们的开发栈中使用Swarm,并希望更好地理解技术背后的动机和设计决策。我们邀请了研究人员、学者和去中心化专家来检查我们的推理,并审核Swarm整体设计的一致性。核心开发人员和构建组件、工具或客户端实现的更广泛生态系统的开发人员应该受益于我们提供的具体规范,以及对其背后思想的解释。

\section*{结构的书\statusgreen}

这本书有三个主要部分。前奏(\ref{part:preface})通过描述历史背景来解释动机,为公平的数据经济奠定了基础。然后我们展示了Swarm愿景。

第二部分,设计与体系结构(\ref{part:designarchitecture}),详细阐述了Swarm的设计与体系结构。本部分涵盖了与Swarm核心功能相关的所有领域。

第三部分,规范(\ref{part:specifications}),提供了组件的正式规范。本文旨在作为Swarm客户端开发者的参考手册。

索引,术语和首字母缩略词的术语表,和附录(\ref{part:appendix})包含正式的论点(\ref{sec:formalisation}),完成纲要。

\section*{如何使用这本书\statusgreen}

前两部分——序曲和设计与建筑——可以作为一个连续的叙述来阅读。那些希望直接进入技术的人可以从设计和架构部分开始,跳过序曲。

Swarm客户端开发者可以从“规格”部分的任何特定组件规格开始,然后再回到“设计和架构”,只要需要更广泛的上下文,或者对规格中明显的选择的理由感兴趣,就可以通过文本参考。

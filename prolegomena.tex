\chapter{绪论\statusgreen}
\green{}
\section*{目标受众\statusgreen}
本书的主要目的是描述Swarm项目第一阶段的丰硕成果,并为参与到即将到来的阶段的Swarm项目的团队和个人提供一个纲要。

这本书是为有技术倾向的读者准备的,他们有兴趣在他们的开发栈中使用Swarm,并希望更好地理解技术背后的动机和设计决策。我们邀请了研究人员、学者和去中心化专家来检查我们的推论,并审核Swarm整体设计的一致性。核心开发人员和构建组件、工具或客户端实现的更广泛生态系统的开发人员应该受益于我们提供的具体规范,以及对其背后思想的解释。

\section*{书的结构\statusgreen}

这本书有三个主要部分。前言(\ref{part:preface})通过描述历史背景来解释动机,为公平的数据经济奠定了基础。然后我们展示了Swarm愿景。

第二部分,设计与架构(\ref{part:designarchitecture}),详细阐述了Swarm的设计与架构。本部分涵盖了与Swarm核心功能相关的所有领域。

第三部分,规范(\ref{part:specifications}),提供了组件的正式规范。本文旨在作为Swarm客户端开发者的参考手册。

索引,术语和首字母缩略词的术语表,和附录(\ref{part:appendix})包含形式化论证(\ref{sec:formalisation})。

\section*{如何使用这本书\statusgreen}

前两部分——前言和设计与架构——可以作为一个连续的叙述来阅读。希望直接进入技术的人可以从设计和架构部分开始,跳过前言。

Swarm客户端开发者可以从“规格”部分的任何特定组件规格开始,如需要更广泛的背景,或者对规格中明显的选择的理由感兴趣,就可以通过文本内引用回到“设计和架构”。

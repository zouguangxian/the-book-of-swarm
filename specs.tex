
The Swarm specification is presented in four chapters. Chapter \ref{spec:convention} enlists all the conventions relating data types, formats and algorithms using the \lstinline{buzz} language as pseudo-code.
Chapter \ref{spec:protocol} ( Protocols) specifies the wire protocols: message formats, serialisation and encapsulation are hard requirements that are crucial for  cross client compatibility. This chapter benefits from \lstinline{protobuf},%
%
\footnote{\url{https://developers.google.com/protocol-buffers}}
%
which is a generic language-neutral, platform-neutral, extensible mechanism for serializing structured data.
The chapter on Strategies (\ref{spec:strategy}) is meant to present recommended incentive-aligned behaviour.
APIs (\ref{spec:api}) gives a formal specification of the high level interfaces of Swarm. Different client implementations are  to be tested against a standard suite of API tests. The API specifications benefit from \lstinline{OpenAPI},%
%
\footnote{\url{https://swagger.io/}}
%
which is an API description format for REST APIs.

\newpage\phantomsection 
\addcontentsline{toc}{chapter}{List of definitions}
\listoftheorems[ignoreall,show={definition}]

\chapter{Data types and algorithms}\label{spec:convention}

\orange{}

\section{Built-in primitives \statusyellow}\label{spec:format:builtin}
\subsection{加密\statusgreen}\label{spec:format:crypto}

本节描述整个规范中使用的加密原语。它们暴露为buzz内置功能。
这些模块包括哈希、随机数生成、密钥派生、对称和非对称加密(ECIES)、挖掘(即寻找瞬时值)、椭圆曲线密钥生成、数字签名(ECDSA)、Diffie- Hellman共享秘密(ECDH)和柯西-里德-所罗门(CRS)擦除编码。

一些内置加密原语(尤其是sha3哈希和ECDSA ecrecover)复制了以太坊虚拟机的加密功能。这些是在以太坊api调用智能合约的帮助下定义的。这个智能合约只实现了“buzz”的原语,并且只有read方法。

\subsubsection{哈希}

基哈希函数实现了以太坊中使用的kecak256哈希。

\begin{definition}[Hashing]\label{def:hash}
\begin{lstlisting}[language=buzz1]
// /crypto

define function hash @input []byte
    ?and/with @suff
    return segment
as
    ethereum/call "sha3" with @input append= @suff
         on context contracts "buzz" 
\end{lstlisting}
\end{definition}  


\subsubsection{随机数生成}

\begin{definition}[Random number generation]\label{def:rng}
\begin{lstlisting}[language=buzz1]
// /crypto

define function random type 
    return [@type size]byte 

\end{lstlisting}
\end{definition}    


\subsubsection{Scrypt关键推导}

加密密钥派生函数实现了\lstinline{scrypt} \cite{percival2009stronger}。

\begin{definition}[Scrypt key derivation]\label{def:scrypt}
\begin{lstlisting}[language=buzz1]
// /crypto

define type salt as [segment size]byte

define type key as [segment size]byte

// params for scrypt key derivation function
// scrypt.key(password, salt, n, r, p, 32) to generate key

define type kdf
    n int // 262144
    r int // 8
    p int // 1

define function scrypt from @password
    with   salt 
    using  kdf
    return key

\end{lstlisting}
\end{definition}  

\subsubsection{矿业助手}

该模块提供了一个非常简单的helper函数,用于查找一个nonce,当该nonce作为挖掘函数的单个参数时返回true。

\begin{definition}[Mining a nonce]\label{def:mine}
\begin{lstlisting}[language=buzz1]
// /crypto

define type nonce as [segment size]byte

define function mine @f function of nonce return bool
as
    @nonce = random key
    return @nonce if call @f @nonce
    self @f
    
\end{lstlisting}
\end{definition}  

\subsubsection{对称加密}

对称加密在计数器模式下使用32字节块大小的修改块密码。
段密钥是通过将特定于块的加密密钥与计数器散列并再次散列生成的。这第二步是必需的,以便可以有选择地以第三方可证明的方式公开段,而不损害块的其余部分的安全性。

该模块提供保持输入长度的块密码加密。

\begin{definition}[Blockcipher]\label{def:crypt}
\begin{lstlisting}[language=buzz1]
// /crypto

// two-way (en/de)crypt function for segment
define function crypt.segment segment
    with key
    at @i uint8
as
    hash @key and @i        // counter mode 
        hash                // extra hashing
        to @segment length  // chop if needed
        xor @segment        // xor key with segment

// two-way (en/de)crypt function for arbitrary length 
define function crypt @input []byte
    with key
    return [@input length]byte
as
    @segments = @input each segment size    // iterate segments of input
        go crypt.segment at @i++ with @key  // concurrent crypt on segments
    return wait for @segments               // wait for results
        join                                // join (en/de)crypted segments
        
\end{lstlisting}
\end{definition}    

\subsubsection{椭圆曲线密钥}

公钥加密与以太坊相同,使用secp256k1椭圆曲线。  


\begin{definition}[Elliptic curve key generation]\label{def:ec-keys}
\begin{lstlisting}[language=buzz1]
// /crypto
define type pubkey  as [64]byte
define type keypair
    privkey [32]byte
    pubkey
    
define type address as [20]byte

define function address pubkey
    return address
as 
    hash pubkey 
        from 12

define function generate 
p    ?using entropy
as
    @entropy = random segment if no @entropy
    http/get "signer/generate?entropy=" append @entropy 
        as keypair

\end{lstlisting}
\end{definition}    

\subsubsection{非对称加密}

非对称加密实现基于secp256k1椭圆曲线的ECIES。 
%  TODO: this needs more detail

\begin{definition}[Asymmetric encryption]\label{def:asymmetric-encryption}
\begin{lstlisting}[language=buzz1]
// /crypto

define function encrypt @input []byte 
    for pubkey
    return [@input length]byte

define function decrypt @input []byte 
    with keypair
    return [@input length]byte

\end{lstlisting}
\end{definition}  


\subsubsection{签名}

Crypto的内置签名模块实现了基于secp2156k1椭圆曲线的ECDSA。实际的签名发生在作为独立进程运行的外部签名者中(可能在安全区域内)。按照以太坊的惯例,签名使用r/s/v格式表示和序列化,

\begin{definition}[Signature]\label{def:signature}
\begin{lstlisting}[language=buzz1]
// /crypto

define type signature
    r segment
    s segment
    v uint8
    signer private keypair
    

define type doc 
    preamble []byte
    context  []byte
    asset    segment
    
define function sign @input []byte 
    by keypair
    return signature
as
    @doc = doc{ "swarm signature", context caller, @input }
    @sig = http/get "signer/sign?text=" append @doc 
        append "&account=" append @keypair pubkey address
            as signature 
    @sig signer = @keypair
    @sig
    
define function recover signature
    with @input []byte
    from @caller []byte
    return pubkey
as
    @doc =  doc{ "swarm signature", @caller, @input } as bytes
    ethereum/call "ecrecover" with 
        on context contracts "buzz" 
            as pubkey

\end{lstlisting}
\end{definition}  


\subsubsection{Diffie-Hellmann共享密钥}

共享秘密模块使用常见的secp256k1椭圆曲线实现基于椭圆曲线的Diffie—Helmann共享秘密(ECDH)。
实际的DH来自外部签名者,然后用盐散列在一起。

\begin{definition}[Shared secret]\label{def:dh}
\begin{lstlisting}[language=buzz1]
// /crypto

define function shared.secret between keypair
    and pubkey
    using salt
    return [segment size]byte
as
    http/get "signer/dh?pubkey=" append @pubkey append "&account=" @keypair address
        hash with @salt

\end{lstlisting}
\end{definition}  

\subsubsection{擦除编码}\label{spec:format:erasure}

Erasure编码接口为直接工作在块列表上的编码器/解码器提供了包装\lstinline{extend/repair}。%
%
\footnote{柯西-里德-所罗门擦除码基于\url{https://github.com/klauspost/reedsolomon}。
}

假设$n$退出$m$编码。
\lstinline{extend}接受$n$数据块的列表和一个表示所需对等数的参数。它只返回奇偶校验块。
\lstinline{repair}接受一个$m$块列表(使用\emph{所有}奇偶扩展)和一个$p=m-n$奇偶数的参数,该参数将最后一个$p$块指定为奇偶块。它只返回$n$修复的数据块列表。
编码器不知道哪些部分是无效的,因此应该在参数中将缺失或无效的块设置为\lstinline{nil}进行修复。
如果需要修复奇偶校验块,可以调用\lstinline{repair @chunks with @parities; extend with @parities}

\begin{definition}[CRS erasure code interface definition]\label{def:crs}
\begin{lstlisting}[language=buzz1]
// /crypto/crs

define function extend @chunks []chunk 
    with @parities uint
    return [@parities]chunk

define function repair @chunks []chunk   
    with @parities uint
   return [@chunks length - @parities]chunk

\end{lstlisting}
\end{definition}

\begin{definition}[CRS erasure coding parameters]\label{def:crs-params}
\begin{lstlisting}[language=buzz1]
// /crypto/crs
define strategy as "race"|"fallback"|"disabled"

define type params 
    parities uint
    strategy 
     
\end{lstlisting}
\end{definition}


\subsection{状态存储\statusgreen}\label{spec:format:statestore}
% \input{specs/api/statestore.tex}

\begin{definition}[State store]\label{def:state-store}
\begin{lstlisting}[language=buzz1]
// /statestore

define type key []byte
define type db  []byte
define type value []byte

define function create db

define function destroy db

define function put value
    to db
    on key
    
define function get key 
    from db
    return value
\end{lstlisting}
\end{definition}


\subsection{本地上下文和配置\statusgreen}\label{spec:format:local}
% \input{specs/api/local.tex}


\begin{definition}[Context]\label{def:scontext}
\begin{lstlisting}[language=buzz1]
// /context

define type contract as "buzz"|"chequebook"|"postage"|""

define type context
    contracts [contract]ethereum/address 
\end{lstlisting}
\end{definition}

\section{Bzz address\statusgreen}\label{spec:format:bzzaddress}
\subsection{包裹地址\statusyellow}

Swarm的Overlay网络使用32字节的地址。为了帮助地址空间的统一利用,这些地址必须使用哈希函数派生。Swarm节点必须关联\gloss{Swarm base account}或\gloss{bzz account}, \gloss{Swarm base account}或\gloss{bzz account}是一个节点(操作符)必须拥有其私钥的以太坊帐户。节点的Overlay地址由该帐户的公钥派生。 

\begin{definition}[Swarm overlay address of node A]\label{def:overlay}
\begin{equation}
\mathit{overlayAddress}(A) \defeq \mathit{Hash}(\mathit{ethAddress}|\mathit{bzzNetworkID})         
\end{equation}
where
\begin{itemize}
\item $\mathit{Hash}$ is the 256-bit Keccak SHA3 hash function
\item \emph{ethAddress} - the ethereum address  (bytes,  not hex) derived from the node's base account public key: $\mathit{account}\defeq\mathit{PubKey}(K_A^\mathit{bzz})[12:32]$), where
    \begin{itemize}
    \item \emph{PubKey} is the \emph{uncompressed} form of the public key of a keypair \emph{including} its $04$ (uncompressed) prefix.
    \item $K_A^\mathit{bzz}$ refers to the node's bzz account key pair
    \end{itemize}
\item \emph{bzzNetworkID} is the bzz network id of the swarm network serialised as a little-endian binary \emph{uint64}.
\end{itemize}
\end{definition}

从某种程度上说,从公钥获得节点地址意味着Overlay地址空间不是免费可用的:要占用一个地址,必须拥有其他节点需要验证的地址的私钥。使用共享的共识文本块的数字签名,这样的身份验证很容易,参见\ref{spec:format:bzzaddress}。

\subsection{衬底地址\statusyellow}

为了使对等点定位网络上的一个节点,Overlay地址与一个底层地址配对。底层地址是节点在底层传输层上的网络位置的字符串表示形式。节点使用它来拨号其他节点来建立peer - to - peer连接。 

\begin{definition}[Swarm underlay multiaddress]\label{def:underlay}
\begin{lstlisting}[]

\end{lstlisting}
\end{definition}


\subsection{BZZ地址\statusyellow}

Bzz地址在功能上是Overlay地址和底层地址的配对。为了确保Overlay地址来自于节点拥有的帐户,以及可验证地证明一个节点可以被调用的底层地址,bzz地址以以下传输格式通信:

\begin{definition}[Swarm bzz address transfer format]\label{def:bzz-address}
\begin{lstlisting}[]
// ID: /swarm/handshake/1.0.0/bzzaddress

syntax = "proto3";

message BzzAddress {
    bytes Underlay = 1;
    Signature Sig  = 2;
    bytes Overlay  = 3; 
}
\end{lstlisting}
\end{definition}

在这里,签名是为了证明一个网络的Overlay层和一个底层地址的关联。 

\begin{definition}[Signed underlay address of node A]\label{def:signed-underlay}
\begin{equation}
\mathit{signedUnderlay}(A) \defeq \mathit{Sign}(\mathit{underlay}|
\mathit{overlay}|\mathit{bzzNetworkID})         
\end{equation}
\end{definition}

使用明文附加的Overlay层和bzz网络ID,并将生成的公钥与bzz网络ID散列。 

\begin{definition}[Node addresses: overlay, underlay,  bzz address]\label{def:bzz-types}
\begin{lstlisting}[language=buzz1]
// /bzz

define type overlay as [segment size]byte
define type underlay []byte

define function overlay.address of pubkey 
    within @network uint64 
as
    hash @pubkey address and @network 
        as overlay

define function valid bzz.address 
    within @network uint64
as
    assert @bzz.address overlay == overlay.address of crypto/recover from @bzz.address signature with @bzz.address @underlay
        within @network
        
define function bzz.address of overlay
    from underlay 
    by @account ethereum/address
as
    @sig = crypto/sign @underlay by @account
    bzz.address{ @overlay, @underlay, @sig }

\end{lstlisting}
\end{definition}




In order to get the overlay address from the transfer format peer info, one recovers from signature the peer's base account public key using the plaintext that is constructed as per \ref{def:signed-underlay}. From the public key, the overlay can be calculated as in \ref{def:overlay}. The overlay address thus obtained needs to be checked against the one provided in the handshake.

Signing of the underlay enables preflight authentication of the underlay of a trusted but not connected node. 

Since underlays are meant to be volatile, we can assume and in fact expect multiple underlays signed by the same node. However, these are meant to be temporally ordered. So one with a newer timestamp invalidates the older one. 

In order to make sure that the node connected through that underlay does indeed operate the overlay address, its authentication must be obtained through the peer connection that was initiated by dialing the underlay. The This protects against malicious impersonation of a trusted overlay potentially. 

\section{Chunks, encryption and addressing\statusyellow}
\subsection{Content addressed chunks \statusgreen}\label{spec:format:chunks}
首先,让我们定义一些基本类型,如\emph{载荷、跨段}。这些固定长度的字节片支持基本单元(如\lstinline{segment size}或\lstinline{payload size})的详细表达式。

\begin{definition}[segment, payload, span, branches]\label{def:chunk-constants}
\begin{lstlisting}[language=buzz1]
// /chunk

define type segment as [32]byte    // unit for type definitions
define type payload as [:4096]byte // variable length max 4Kilobyte
define type span as uint64         // little endian binary marshalled

define function branches 
    as payload size / segment size

\end{lstlisting}
\end{definition}

现在我们来看看\emph{地址,关键}和\emph{参考}的定义:                                             

\begin{definition}[Chunk reference]\label{def:chunk-reference}
\begin{lstlisting}[language=buzz1]
// /chunk

define type address as [segment size]byte

define type reference 
    address                   // result of bmt hash
    key if context.encryption // decryption key optional (context dependent)

\end{lstlisting}
\end{definition}

现在,将chunk定义为具有跨度和有效载荷的对象。

\begin{definition}[Content addressed chunk]\label{def:chunks}
\begin{lstlisting}[language=buzz1]
// /chunk

define type chunk
    span      // length of data span subsumed under node
    payload   // max 4096 bytes 

define function address of chunk
as
    @chunk payload bmt/hash with @chunk span 

define function create from payload
    ?over span
as
    @span = @payload length if no @span
    @chunk = chunk{ @span, @payload }
    return @chunk if no context encryption
    @key = encryption.key for @chunk 
    @chunk encrypt with @key
\end{lstlisting}
\end{definition}

其中length是length字段的内容,reference size是引用哈希值和解密密钥的大小之和,解密密钥目前是64,因为我们使用的是256位哈希值和256位密钥。

以便在返回明文块之前删除解密后的填充。 

\begin{definition}[Span to payload length]\label{def:span}
\begin{lstlisting}[language=buzz1]
// /chunk

define function payload.length of span
as
    while @span >= 4096 
        @span = @span + 4095
           / 4096
           * reference size
    return @span
\end{lstlisting}
\end{definition}

最后,我们可以定义用于检索和存储的块的公共API。

\begin{definition}[Chunk API retrieval]\label{def:retrieve}
\begin{lstlisting}[language=buzz1]
// /chunk

define function retrieve reference
has api GET on "chunk/<reference>"
as 
    retrieve @reference address as chunk
        (decrypt with @reference key if @reference key)
\end{lstlisting}
\end{definition}


\begin{definition}[Chunk API: storage] \label{def:store}
\begin{lstlisting}[language=buzz1]
// /chunk


define function store payload
    ?over span 
has api POST on "chunk/(?span=<span>)"
    from payload as body
as 
    @chunk = create from @payload over @span
    reference{ @chunk address, @chunk key }
\end{lstlisting}
\end{definition}



\subsection{Single owner chunk \statusgreen}\label{spec:format:soc}
单所有者块是群中的第二种类型块(参见\ref{sec:single-owner-chunks})。它们构成了蜂群饲料的基础。


\begin{definition}[Single owner chunks]\label{def:soc}
\begin{lstlisting}[language=buzz1]
// /soc

// data structure for single owner chunk
define type soc 
    id         [segment size]byte // id 'within' owner namespace
    signature  crypto/signature   // owner attests to <content, id> 
    chunk                         // content: embeds a content chunk

// constructor for single owner chunks
define function create from chunk 
    by @owner crypto/keypair
    on @id [segment size]byte
 as
    @sig = crypto/sign @id and @chunk address by @owner
    soc{ @id, @sig, @chunk }

define function address soc
as
    hash @soc id and @soc signature signer address
    
\end{lstlisting}
\end{definition}

\begin{definition}[Single owner chunk API: retrieval]\label{def:soc-retrieve}
\begin{lstlisting}[language=buzz1]
// /soc

define function retrieve @id [segment size]byte
    by @owner ethereum/address
    ?with key
has api GET on "soc/<owner>/<id>(?key=<key>)"
as
    retrieve hash @id and @owner 
        as soc
            chunk (decrypt with @key if @key)
\end{lstlisting}
\end{definition}


\begin{definition}[Single owner chunk API: storage]\label{def:soc-store}
\begin{lstlisting}[language=buzz1]
// /soc

define function store payload
    on @id [segment size]byte
    by @owner ethereum/address 
    ?over span
has api POST on "soc/<owner>/<id>?span=<span>&encrypt=<encrypt>"
    from <payload> as body
as 
    @span = @payload length if no @span
    @chunk = chunk{ @span, @payload }
    if context encryption then
        @key = encryption.key for @chunk 
        @chunk encrypt= with @key
    @soc = create from @chunk on @id by private key of @owner
    reference{ @soc store address, @key }
             
\end{lstlisting}
\end{definition}

\subsection{Binary Merkle Tree Hash \statusyellow}\label{spec:format:bmt}
用于获取默认内容寻址块地址的散列方法称为\gloss{binary Merkle tree hash},或简称为\gloss{BMT hash}。 

\subsubsection{计算BMT哈希值}

二叉树的基段是块内容数据的子序列。 
段的大小是32字节,这是用于构造树的\emph{基地哈希}的摘要大小。 
给定用于表示文件的Swarm哈希树(参见afaed),假设中间块包括了对其他块的引用。 

获取序列的BMT哈希包括以下步骤:

\begin{enumerate}
\item \emph{填充} -如果内容小于最大块大小(4096字节,\ref{sec:content-addressed-chunks}),则用0填充直到块大小。注意,这个零填充仅用于哈希,不会影响块数据大小。
\item \emph{块数据层} -计算填充块中\emph{双段}的基哈希值,即数据的段大小($2 * 32$)单元,并将结果连接起来。
\item \emph{构建树} -对结果重复上一步,直到结果只是一个部分。
\item \emph{计算跨度} -计算数据的跨度,即包含在由未填充数据表示为64位小端整数值(参见afaed)的块下的数据的大小。            
\item \emph{完整性保护} -将span添加到二叉树的根哈希中,并计算数据的\emph{基地哈希}。
\end{enumerate}

\begin{definition}[BMT hash]\label{def:bmt-hash}
\begin{lstlisting}[language=buzz1]
// /bmt

define function hash payload 
    with span
as
    @padded = @payload as [:chunk size]byte    // use zero padding 
    // for BMT hashing only
    hash @span and root of @padded over chunk size 
    
define function root of @section []byte
    over @len uint
as
    return hash @section        // data level
        if @len == 2 * segment size
    @len /= 2                  // recursive call
    @children = @section each @len go self over @len
    wait for @children 
        join hash
    

\end{lstlisting}
\end{definition}

\subsubsection{包含证明}

将片段与这些块中打包的哈希对齐,可以将包含证明的概念扩展到文件中。
BMT哈希支持紧凑的第三方可验证的段包含证明。

\red{指定}
\subsection{Encryption \statusyellow}\label{spec:format:encryption}

Swarm中的对称加密在counter模式中使用了一个稍微修改过的块密码版本。

如果给定了块的加密种子,则该加密种子来自主种子,否则只是随机生成的。 

对单个块(和整个内容)的引用是加密数据的散列和解密密钥的连接(请参阅\ref{def:chunk-reference})。这意味着加密的Swarm引用(64字节)将比未加密的Swarm引用(32字节)长。当一个节点同步加密块时,它不会以任何方式与其他节点共享完整的引用(或解密密钥)。因此,其他节点将无法访问原始数据,甚至无法检测一个chunk是否被加密。

\begin{definition}[Chunk encryption/decryption API]\label{def:encrypt}
\begin{lstlisting}[language=buzz1]

// generate key for a chunk
define encryption.key for chunk
    ?with @seed [segment.size]byte
as
    return crypto/random key if no @seed // generate new 
    hash @seed and @chunk address

define function encrypt chunk
    with key 
as   
    @segments = @chunk data pad to chunk size
        each segment size 
            go crypt at @i++ with @key 
    @span = chunk span crypt at branches with @seed 
    @payload = wait for @segments 
        join
    chunk{ @span, @payload } 


define function decrypt chunk
    with key 
as       
    @span  = @chunk span 
        crypt at branches with @key 
    @segments = chunk data to @span payload.length
        each segment size 
            go crypt at @i++ with @key
    @payload = wait for @segments 
        join
    chunk{ @span, @payload } 

\end{lstlisting}
\end{definition} 
       
        
加密的Swarm块与明文块没有区别,因此不需要在P2P协议级别上更改以适应它们。提议的加密方案是端到端,这意味着加密和解密是在端点上完成的,即http代理层运行的地方。这有一个重要的后果,即公共网关不能用于加密内容。另一方面,蜂房模块化设计允许客户端在外部api上加密,同时通过网关代理所有其他调用。



\section{Files, manifests and data structures\statusyellow}\label{spec:format:data-structures}
\subsection{Files and the Swarm hash \statusyellow}\label{spec:format:files}

这个表给出了块跨度表示的数据大小的概述,这取决于递归的级别。

\begin{table}[ht]
\begin{tabular}{|r||r|r|r||r|r|r|}
\hline
&\multicolumn{6}{|c|}{span}\\\hline
&\multicolumn{3}{|c|}{unencrypted}
&\multicolumn{3}{|c|}{encrypted}\\\hline
level & chunks & $\mathit{log}_2$ of bytes & standard & chunks & $\mathit{log}_2$ of bytes & standard \\
\hline\hline
0 & 1 & 12 & 4KB & 1 & 12 & 4KB \\\hline
1 & 128 & 19 & 512KB & 64 & 18 & 256KB \\\hline
2 & 16,384 & 26 & 67MB & 4,096 & 24 & 16MB \\\hline
3 & 2,097,152 &33 & 8.5GB & 262,144 &  30 & 1.07GB\\\hline
4 & 268.44M & 40  & 1.1TB & 16.78M & 36 & 68.7GB\\\hline
5 & 34,359.74M & 47 & 140TB & 1,073.74M & 42  & 4.4TB\\\hline
\end{tabular}
\caption{Size of chunk spans}
\end{table}




\subsubsection{计算群哈希}

客户端自定义冗余通过CRS擦除编码实现(参见\ref{sec:erasure}和\ref{sec:features});使用它需要一些CRS参数。
      
\begin{definition}[File API: upload/storage;  swarm hash]\label{def:swarm-hash}
\begin{lstlisting}[language=buzz1]
// /file

define function encode @levels []chunk stream 
    for @level uint
as
    @chunks = @levels at @level     // read chunk stream 
    @crs = context crs              // 
    @m = branches (- @crs parities if @crs) 
    
    @parent = read @m from @chunks  // read up to m chunks from stream blocking
        (append crs/extend with @crs parities if @crs)
            each chunk/store        // package children reference
                join as chunk
                
    if @levels length == @level+1 then
        @levels append= stream{}
        go self @levels for @level+1
    
    write @parent to @levels at @level+1 
    if no @chunks then
        close @levels at @level + 1
    else
        self @chunks for @level
            
define function split @data byte stream 
as
    @level = chunk stream{}
    go @data each chunk size as chunk 
        write to @chunks
    @level

define function upload byte stream as @data
has api POST  on "file/" from @data as body
as
    @levels append= @data split     // 
    go encode @levels for 0
    @top = @levels each wait for  // wait for all levels to close
    return @top at 0              // return root hash as address        
\end{lstlisting}
\end{definition}
                   

\begin{definition}[File   API: download/retrieval]\label{def:file-retrieval}
\begin{lstlisting}[language=buzz1]
// /file

define function copy to @reader stream of byte{}
    from @chunks stream of []chunk
    using @buffers stream of [@buffer.size]stream of [branches]chunk
as
    @chunks = read @buffers if no @chunks
    @chunk = read @chunks
    if no @chunk
        write @chunks to @buffer
        self @reader using @buffers
    write @chunk data to @reader

define function download reference
    ?using @buffer.size uint64
has api GET  on "file/<reference>" 
as
    @reader = stream of byte{} 
    @buffers = stream of [@buffer.size]stream of [branches]chunk
    chunk/retrieve @reference            // root chunk retrieval
        go decode into @buffer down to 1 // traverse          
    copy into @reader from @buffers 

define function decode chunk 
    into @response chunk stream
    ?down to @limit uint8
as
     
    @crs = context crs
    @all = @m = branches 
    if @crs then
        @m -= @crs parities
        if @crs strategy is not "race" then 
            @all = @m 
        
    @chunks = @chunk segments up to @all 
        each go as reference retrieve 
    wait for @m in @chunks 
    
    if @crs then 
        cancel @chunks
        @chunks = @chunks crs/repair with @crs parities 
    
    if @chunk span < chunk size exp @limit + 1 then
        @chunks each  into @reader
        
       
    @chunks each go self down to @limit  

\end{lstlisting}
\end{definition}


\begin{definition}[File info]\label{def:file-info}
\begin{lstlisting}[language=buzz1]
// /file

define type info
    mode        int64 
    size        int64
    modified    time        

\end{lstlisting}
\end{definition}

\subsection{Manifests \statusyellow}\label{spec:format:manifests}
清单表示字符串到引用的映射(参见\ref{sec:collections})。其主要目的是实现文件集合(网站)的Swarm,并支持基于url的内容寻址。本节定义与清单相关的数据结构,以及实现清单API的查找和更新算法(参见\ref{sec:manifests-ux}和\ref{spec:api:manifest})。

清单条目可以被理解为关于文件的元数据,该文件指向该文件并可通过其引用进行检索(参见afaed)。元数据是相当多样化,从访问控制所需信息,类似于一个给定的文件系统、文件信息擦除编码所需的信息(见\ref{sec:headers}和\ref{spec:format:erasure} \ref{sec:erasure}),浏览器的信息,也就是说,响应头如内容类型(MIME信息)和最重要的是对文件的引用。访问控制举例说明了将清单用作简单的键值存储(请参阅\ref{sec:access-control}、\ref{sec:access-control-ux}进行讨论,并参阅\ref{spec:format:access-control}和\ref{spec:api:access-control}了解规范)。

\begin{definition}[Manifest entry]\label{def:manifest-entry}
\begin{lstlisting}[language=buzz1]
// /manifest

// manifest entry encodes attributes 
define type entry 
    file/info              // FS file/dir info
    access/params          // access control params
    crs/params             // erasure coding - CRS params
    reference              // reference
    headers                // http response headers 

define type headers
    content.type  [segment size]byte

\end{lstlisting}
\end{definition}

\begin{definition}[Manifest data structure]\label{def:manifests}
\begin{lstlisting}[language=buzz1]
// /manifest

define type node 
    entry  *entry          // reference to chunk serialised as entry
    forks  [<<256]fork     // sparse array of max 256 fork

// fork encodes a branch
define type fork 
    prefix segment   // compaction 
    node   *node     // reference to chunk serialised as node

\end{lstlisting}
\end{definition}

\begin{definition}[Manifest API: path lookup]\label{def:manifests-lookup}
\begin{lstlisting}[language=buzz1]
// /manifest

define function lookup @path []byte
    in *node
has api GET on "/manifest/<@node:reference>/<@path>"
as 
    context access = @node entry access/params 
    // manifest is  a compacted trie
    @fork = @node forks at head @path 
    // if @path empty, the  paths matched return the entry
    if no @path then 
        return @node entry

    if @fork prefix is prefix of @path then // including == 
      return self @path from @fork prefix length
          in @fork node 
   fail with "not found"

\end{lstlisting}
\end{definition}


\begin{definition}[Manifest API: update]\label{def:manifest-update}
\begin{lstlisting}[language=buzz1]
// /manifest

define function add *entry  
    to *node 
    on @path []byte 
has api PUT on "/manifest/<@node>"
    
as
    // if called on nil call on zero value
    @node = node{} if no @node 

    // if empty path then change entry field of node
    if no @path then
        @node entry = @entry
        return store @node

    // lookup the fork based on the first byte of path
    @fork = @node forks at head @path
    // if no fork yet, add the singleton node 
    if no @fork then
        @node forks at head @path =
            fork{@path, store node{@entry}}
        return store @node

    @common = prefix of @path and @fork prefix // common cannot be empty
    @rest = @fork prefix from @common length
    @newnode = node{}
    @newnode forks at head @rest = fork{@rest, @fork node}
    @midnode = self @entry to @newnode on @path from @common length 
    @node forks at head @path = fork{ @common, @midnode } 
    @node store
    

\end{lstlisting}
\end{definition}

\begin{definition}[Manifest API: remove]\label{def:manifest-remove}
\begin{lstlisting}[language=buzz1]
// /manifest

define function remove @path []byte 
    from *node
has api DELETE on "/manifest/<@node>/<@path>"
as
    // if called on nil call on zero value
    return node{} if no @node 

    // if empty path then change entry field of node
    if no @path then
        return nil if @node forks length == 0 
        @node entry = nil //entry exists
        return store @node

    // lookup the fork based on the first byte of path
    @fork = @node forks at head @path
    // if no fork yet, add the singleton node 
    return @node if no @fork

    @common = prefix of @path and @fork prefix // common cannot be empty
    return @node if @common and @fork prefix have different length          // path not found
    
    @rest = @fork prefix from @common length
    @newnode =  self @rest from @fork node               
    if no @newnode then           // deleted item was terminal node, delete fork 
        @node forks at head @res = nil
    else if @newnode forks length == 1 then // compact non-forking nodes 
        @singleton = @newnode forks first
        @newprefix = @common append @singleton prefix
        @node forks at head @path = 
            fork{ @newprefix, @singleton node }
    else
        @node fork at head @path node = @newnode
        
    @node store
    

\end{lstlisting}
\end{definition}


\begin{definition}[Manifest API:  merge]\label{def:manifest-merge}
\begin{lstlisting}[language=buzz1]
// /manifest

define function merge @new *node  
    to  @old *node 
has api POST on "/manifest/<@old:reference>/<@new:reference>"
as
    // if called on nil call on zero value
    return @new if no @old
    return @old if no @new
    @node = node{ @new or @old } 
    @new forks pos or @old forks pos 
        each bit @pos go 
            @fork = merge.fork @new forks at @pos
                to @old forks at @pos
            @node forks at @fork prefix head = @fork 
    @node store 

define function merge.fork @new fork 
    to @old fork    
as
    @common = prefix of @new prefix and @old prefix 
    @restnew = @new prefix from @common length
    @restold = @old prefix from @common length
    if no @restnew and no @restold then  
        return fork{@common, merge.node @new reference to @old reference}
    @node = add @new reference to nil on @restnew 
        add to @old reference @restold
    fork{ @common, @node }
         
\end{lstlisting}
\end{definition}

% \section{Entanglement coding \statusred}\label{spec:format:entanglements}
%\input{specs/format/entanglement.tex}
% composite api, resolver, tags
\subsection{解析器}\label{spec:format:resolver}


\begin{definition}[Resolver]\label{def:resolver}
\begin{lstlisting}[language=buzz1]
// /resolver

define type resolver 
    api url
    address ethereum/address
    tlds []string 

define function resolve @host []byte through @resolver
as
    @tld = @host split on '.'' last    @resolver = @resolvers any tlds any ==  @tld
        
    ethereum/call "getContentHash" of @host using @resolver api at @resolver address
        as address
                     
\end{lstlisting}
\end{definition}




\subsection{锁住}\label{spec:format:pinning}


\begin{definition}[Pinning]\label{def:pinning}
\begin{lstlisting}[language=buzz1]
// /pin

define type pin
    reference
    chunks uint64
    
define function list
    return []pin
has api GET "/pin/"

define function view reference
    return uint64
has api GET "/pin/<reference>"
    
define function pin reference
    return uint64
has api PUT "/pin/<reference>"

define function pin reference
    return uint64
has api DELETE "/pin/<reference>"
    
       
\end{lstlisting}
\end{definition}



\subsection{标签}\label{spec:format:tags}


\begin{definition}[Tags]\label{def:tags}
\begin{lstlisting}[language=buzz1]
// /tag

define type tag
    id     [segment size]byte
    reference     // the current root
    complete bool // if local upload finished
    total  uint64 // number of chunks expected
    split  uint64 // number of chunks split
    stored uint64 // number of chunks stored locally
    seen   uint64 // number of chunks already in db
    sent   uint64 // number of chunks sent with push-sync
    synced uint64 // number of chunks synced


define function list
    return []tag
has api GET "/tag/"

define function view reference
    return uint64
has api GET "/tag/<reference>"
    
define function add reference
    return uint64
has api POST "/tag/"

define function remove reference
    return uint64
has api DELETE "/tag/<reference>"
    
       
\end{lstlisting}
\end{definition}

\subsection{存储}\label{spec:format:bzz-api}


\begin{definition}[Public storage API]\label{def:bzz}
\begin{lstlisting}[language=buzz1]
// /bzz

define function upload @data stream of byte
has api POST on "bzz:/<host>/<path>" from @data as body
as
    @root = resolver/resolve @host
    @manifest = access/unlock @root
    @entry = file/upload @data
    @reference =  chunk/store @entry as []byte
    manifest/add @reference to @manifest on @path


define function download @path []byte     from @host []byte
has api POST on "bzz:/<host>/<path>" from @data as body
as
    @root = resolver/resolve @host
    @manifest = access/unlock @root
    @entry = manifest/lookup @path in @manifest
    file/download @entry reference
\end{lstlisting}
\end{definition}


\section{Access Control \statusgreen}\label{spec:format:access-control}

\begin{apiRoute}{POST}{/access/\param{address} }{Lock ACT for address as in \ref{def:ac-api}}{
}{ }

\begin{routeParameter} 
\routeParamItem{address}{hex string}
\end{routeParameter}
\begin{routeResponse}{application/json}
\responseItem{201}{Created}{root access manifest reference in response body}
\responseItem{400}{Bad Request}{Encrypted content but no decryption key in reference}
\responseItem{403}{Forbidden}{Encrypted content but no decryption key in reference}
\responseItem{404}{Not found}{}
\responseItem{408}{Request Timeout}{Timeout retrieving referenced manifest}
\responseItem{420}{Enhance your calm}{Recovery initiated but request timed out}
\end{routeResponse}
\end{apiRoute}



\begin{apiRoute}{GET}{/access/\param{address} }{Unlock ACT for address as in \ref{def:ac-api}}{
}{ }

\begin{routeParameter} 
\routeParamItem{address}{hex string}
\end{routeParameter}
\begin{routeResponse}{application/json}
\responseItem{200}{ok}{}
\responseItem{400}{Bad Request}{Address not well formed}
\responseItem{401}{Unauthorized}{Access denied: AC unlock failed}
\responseItem{403}{Forbidden}{Encrypted content but no decryption key in reference}
\responseItem{408}{Request Timeout}{Timeout retrieving referenced manifest}
\responseItem{420}{Enhance your calm}{Recovery initiated but request timed out}\end{routeResponse}
\end{apiRoute}




\begin{apiRoute}{PUT}{/access/\param{root}/\param{pubkey}}{Add entry for pubkey to the  ACT referred in the root access manifest \ref{def:act-api}}{
}{ }

\begin{routeParameter} 
\routeParamItem{root}{hex string - reference to root access manifest}
\routeParamItem{pubkey}{hex string - public key of grantee}
\end{routeParameter}
\begin{routeResponse}{application/json}
\responseItem{201}{Created}{Reference to new manifest root in response body}
\responseItem{400}{Bad Request}{Address or public key not well formed}
\responseItem{401}{Unauthorized}{Permission denied: creating session key failed}
\responseItem{403}{Forbidden}{Encrypted content but no decryption key in reference}
\responseItem{408}{Request Timeout}{Timeout retrieving referenced manifest}
\responseItem{420}{Enhance your calm}{Recovery initiated but request timed out}
\end{routeResponse}
\end{apiRoute}


\begin{apiRoute}{DELETE}{/access/\param{root}/\param{pubkey}}{Remove entry for pubkey from ACT referred in the root access manifest, see \ref{def:act-api}}{
}{ }

\begin{routeParameter} 
\routeParamItem{root}{hex string - reference to root access manifest}
\routeParamItem{pubkey}{hex string - public key of grantee}
\end{routeParameter}
\begin{routeResponse}{application/json}
\responseItem{201}{Created}{Reference to new manifest root in response body}
\responseItem{400}{Bad Request}{Address or public key not well formed}
\responseItem{401}{Unauthorized}{Permission denied: creating session key failed}
\responseItem{403}{Forbidden}{Encrypted content but no decryption key in reference}
\responseItem{408}{Request Timeout}{Timeout retrieving referenced manifest}
\responseItem{420}{Enhance your calm}{Recovery initiated but request timed out}
\end{routeResponse}
\end{apiRoute}


\section{PSS \statusyellow}

\subsection{PSS message\statusgreen}
\label{spec:format:pss-messsage}
\subsection{直接pss消息与木马块}

Pss有两种基本类型,一个消息和一个木马块结构,它封装加密的序列化消息,并包含一个nonce,该nonce被挖掘来使结果块的内容地址(BMT哈希)匹配目标。


\begin{definition}[Basic types: topic, targets, recipient, message and trojan]\label{def:pss-message}
\begin{lstlisting}[language=buzz1]
// /pss


define type topic        as [segment size]byte       // obfuscated topic matcher
define type targets      as [][]byte       // overlay prefixes 
define type recipient    as crypto/pubkey

// pss message
define type message 
    seal    segment            
    payload [!:4030]byte    // varlength padded to 4030B
    
// trojan chunk
define type trojan 
    nonce   segment           // the nonce to mine 
    message [4064]byte        // encrypted msg 
\end{lstlisting}
\end{definition}


消息的编码方式允许完整性检查,同时模糊了主题。将有效负载与主题打包的操作称为\emph{密封}


\begin{definition}[Sealing/unsealing the message]\label{def:pss-sealing}
\begin{lstlisting}[language=buzz1]
// /pss

define function seal @payload []byte
    with topic
as
    @seal = hash @payload and @topic // obfuscate topic
        xor @topic          
    return message{ @seal, @payload }

define function unseal message
    with topic 
as
    @seal = hash @message payload and @topic 
    if @topic == @seal xor @message seal then // check 
        return @payload 
    return nil
    
\end{lstlisting}
\end{definition}

函数在消息和木马块之间进行\lstinline{wrap/unwrap}转换。\lstinline{wrap}接受一个可选的接收方公钥来对消息进行非对称加密。
目标是一个覆盖地址前缀列表,它是从接收者的覆盖地址派生出来的,指定长度以保证匹配它的块在推送同步时只会以接收者作为结果。   

\begin{definition}[Wrapping/unwrapping]\label{def:wrap}
\begin{lstlisting}[language=buzz1]
// /pss

define function wrap message 
    for recipient
    to  targets
as 
    @msg = @message 
        (crypto/encrypt for @recipient if @recipient) 

    @nonce = crypto/mine @n such that
        @targets any is prefix of
            trojan{@n, @msg} as chunk address 
    trojan{@nonce, @msg} as chunk 

define function unwrap chunk
    for recipient
as
    @chunk bytes 
        (crypto/decrypt  for @recipient if @recipient)
            as message

\end{lstlisting}
\end{definition}

当数据块到达节点时,存储组件将\lstinline{pss/deliver}作为钩子调用。
首先,使用接收方私钥对消息进行拆封,并使用所有订阅的主题API客户端对消息进行拆封。如果解封成功,则验证消息完整性和主题匹配,因此将有效负载写入为相关主题注册的流中。

\begin{definition}[Incoming message handling]\label{def:delivery}
\begin{lstlisting}[language=buzz1]
// /pss

// mailbox is a handler type, expects payload
// sent sealed with the topic to be delivered via the stream 
define type mailbox
    topic
    deliveries stream of []byte 
    
define context mailboxes as []mailbox

define function deliver chunk
    @msg = @chunk unwrap for context recipient
    mailboxes each @mailbox 
        @payload = unseal @msg  with @mailbox topic
        if @payload then 
            write @msg payload 
                to @mailbox deliveries 
    

\end{lstlisting}
\end{definition}


\begin{definition}[pss API: send]\label{def:send}
\begin{lstlisting}[language=buzz1]
// /pss

define function send @payload []byte
    about topic
    for recipient
    ?to    targets
has api POST on "/pss/<recipient>/<topic>(?targets=<targets>)"
    with @payload as body
as 
    targets = lookup.targets for @recipient if no @targets
    context tag = tag/tag{}
    seal @payload with @topic        // seal with topic
        wrap for @recipient          // encrypt if given recipient
            to @targets              // mine nonce and returns trojan chunk
                store                // to be sent by push-sync
    return tag                       // tag to monitor status 
    
\end{lstlisting}
\end{definition}

\begin{definition}[pss API: receive]\label{def:receive}
\begin{lstlisting}[language=buzz1]
// /pss

define function receive about topic 
    on uint64 @channel
has api POST on "/pss/subscribe/<topic>(?on=<channel>)"
as 
    @stream = open @channel
    context mailboxes append= mailbox{ @topic, @stream }
    
define function cancel topic
    on @channel uint64
has api DELETE on "/pss/subscribe/<topic>(?on=<channel>)"
as
    context mailboxes any ch
\end{lstlisting}
\end{definition}

\subsection{信封}

\begin{definition}[Envelope]\label{def:pss-envelope}
\begin{lstlisting}[language=buzz1]
// /pss

define type envelope
    id  [segment size]byte
    sig crypto/signature
    ps  postage/stamp   
    
\end{lstlisting}
\end{definition}

\subsection{Update notifications \statusred}\label{spec:format:update-notifications}
%\input{specs/format/update-notifications.tex}

\subsection{Chunk recovery  \statusyellow}\label{spec:format:recovery}
\input{specs/format/recovery.tex}

\section{Postage stamps \statusorange}\label{spec:format:postage-stamps}

\subsection{邮票的上传}
\subsubsection{见证类型}

可以有在\emph{付款证明}的结构和语义上不同的邮票实现。为了允许在开发新的加密机制时使用它们,\lstinline{witnessType}参数指示所使用的证人的类型。 

见证类型$0$代表ECDSA见证,它是一个字节片上的ECDSA签名,该签名由1)序言常量2)块哈希3)批处理引用4)有效到date.% 
%
\footnote{ECDSA签名的二进制编码是65字节,由该签名的$r$(32字节)、$s$(32字节)和$v$(1字节)参数串联而成。该签名按照secp256k1椭圆曲线计算,就像以太坊交易的签名一样。}
%
这是邮票合同和客户必须实现的最低要求。%
%
\footnote{就戳戳验证契约所消耗的气体和链外使用的计算资源而言,ECDSA见证是最简单和最便宜的解决方案。此外,除了以太坊严重依赖的加密假设之外,它并不依赖加密假设,因此只要以太坊被认为是加密安全的,加密技术的任何进步都不能使这种见证类型变得不安全。这是证明该证人类型是唯一要执行的强制性证人类型的理由。}

见证类型$1$是RSA见证,它是与上述相同的128字节上的RSA签名。RSA签名的二进制编码是可变长度的,是RSA签名$s$.%的Solidity ABI编码数组
%
\footnote{如在\emph{PKCS \# 1}、\url{https://tools.ietf.org/html/rfc8017}和RSA公钥参数$n$ (RSA模量)和$e$(公共指数)中定义的。}

指定RSA见证,这样就可以以一种简单的方式实现盲戳服务,以减轻链接使用相同私钥签名的块的能力所带来的隐私问题。即使盲ECDSA签名也存在,但它们的协议需要更多的通信轮数,这使得此类服务的实现更加复杂,更容易出错,性能也更差。 

包含整个每个RSA公钥的见证而不是存储在合同的公钥状态和刚从证人引用它通过减少天然气的成本是合理的交互设计与合同以及能否经得住时间的考验,以防在Ethereum租引入合同状态。这些考虑比邮票的简便性更重要,略微降低了上传和转发盖章内容的成本。

请注意,加密技术的进步可以使RSA证人不安全,而不会使以太坊不安全,因此,如果RSA签名的安全性受到威胁,RSA证人可以在协议的未来版本中逐步淘汰。此外,请注意,这种盲签名服务并不是完全不可信的,因为它们可能造成的损害是有限的。基于ZK证明的无Trustless blind stamping服务在现阶段还不可行,目前的算法性能还不够,但考虑到该领域的快速发展,未来可以期待合适的算法的发展。在这种情况下,必须在单独的SWIP中指定相应的见证类型。

\subsubsection{合同的升级}

为了便于在发现漏洞或某些特性扩展(如添加新的证人类型)的情况下升级合同,建议将拥有付款数据库的资金的那部分和核查证人的那部分分别订立合同,以便能够在最少中断的情况下进行向后兼容的升级。

为了避免集中控制,还建议在客户端配置中引用证人验证合同,以便客户端操作人员可以在新合同可用时独立决定何时以及是否切换到新合同。


参与同一邮资系统的节点被配置为在同一区块链上引用同一合同。本合同必须符合以下接口:

\begin{definition}[Postage contract]\label{def:postage}

\end{definition}


如果\lstinline{witness}包含的证明检查出声明中的所有其他参数,则此访问器方法返回\lstinline{true} 
有效期,即\lstinline{block.timestamp} (\lstinline{TIMESTAMP EVM}操作码的输出)介于\lstinline{beginValidity}(含)和 
\lstinline{endValidity}(独家)。在有效期之外,返回值为\lstinline{undefined}。
 

\begin{definition}[Postage stamp basic types: batchID, address, witness, stamp, validity]\label{def:postage-stamp}
\begin{lstlisting}[language=buzz1]
// /postage

define type batchid as [segment size]byte  
define type address as bzz/address
define type witness as crypo/signature

// postage stamp
define type stamp  
    batchid
    address
    witness

define function valid stamp
as 
    // check validity on blockchain
    ethereum call "valid" using context contracts "postage"
        with @stamp
        
\end{lstlisting}
\end{definition}

\subsection{邮资订阅}\label{spec:format:subscriptions}

\subsection{邮资彩票比赛}\label{spec:format:race}
% \section{Honey token and multi-chain support}\label{spec:format:honey}
%\input{specs/format/honey.tex}


\chapter{Protocols}\label{spec:protocol}

\section{Introduction \statusorange}\label{spec:protocol:intro}

\subsection{衬底网络\statusgreen}

\lstinline{libp2p}网络堆栈为\ref{sec:underlay-transport}中布局的群底层网络提供了所有必需的属性。

\begin{enumerate}
\item 以\emph{多地址}的形式为每个节点提供寻址,这里称为底层地址。每个节点可以有多个底层地址,这取决于配置的传输和网络侦听地址。
\item 拨号是通过\lstinline{libp2p}支持的网络传输提供的。
\item 监听由\lstinline{libp2p}支持的网络传输提供。
\item 在两个对等体之间建立实时连接,并保持打开状态以接受或发送消息。
\item 通过\lstinline{TLS}和\lstinline{libp2p secio}流安全传输提供信道安全。
\item 协议复用是由\lstinline{libp2p mplex}流复用协议提供的。
\item 通过使用\lstinline{libp2p}双向流来验证对等端对发送消息的响应,可以提供交付保证。
\item \lstinline{libp2p}并不强制序列化,因为它提供了字节流,允许每个协议灵活地选择最合适的序列化。推荐的序列化方式是\lstinline{Protobuf},在流中使用\lstinline{varint}分隔的消息。
\end{enumerate}

\subsection{协议和流\statusgreen}

对等点之间的通信是在协议中组织的,在一个唯一名称下的逻辑单元,可以定义一个或多个\emph{流}。\lstinline{libp2p}提供流作为通信的基本通道。流是字节的全双工通道,在两个对等体之间的单个连接上多路复用。

每一个流程定义:

\begin{itemize}
\item 以semver形式遵循语义版本控制的版本
\item 数据序列化的定义
\item 通过全双工流在对等体之间传递的数据序列
\end{itemize}

流由\lstinline{libp2p}区分大小写的协议id标识。以下约定用于构造流标识符:

\begin{lstlisting}
/swarm/ProtocolName/ProtocolVersion/StreamName
\end{lstlisting}

\begin{itemize}
\item 所有流id都以\lstinline{/swarm}为前缀。
\item \lstinline{ProtocolName}是标识协议的任意字符串。
\item \lstinline{ProtocolVersion}是一个semver形式的字符串,用于指定协议实现之间随时间的兼容性。
\item \lstinline{StreamName}是一个任意字符串,用于标识定义为协议一部分的流。
\end{itemize}

\subsection{数据交换序列\statusgreen}

在一个打开的流下,数据传递序列必须是同步的。可以同时打开多个流,这些流在同一连接上进行多路复用,独立和异步地交换数据。流可以使用不同的数据交换序列,如:

\begin{itemize}
\item \emph{单独的消息发送} -在关闭流之前不等待对等体的响应,如果它不需要。
\item \emph{多个消息发送}—在关闭流之前不从对等端读取而发送到对等端的一系列数据。
\item \emph{请求/响应}—在关闭流之前需要对单个请求的单个响应。
\item \emph{多个请求/响应周期}——在每个请求之后关闭流之前都需要一个同步响应。
\item \ref{spec:protocol:hive} -需要在单个流上以精确的顺序(参见\ref{spec:protocol:hive}中的握手协议)使用多种消息类型。
\end{itemize}

流有预定义的序列,为了一个目的尽可能地保持简单。对于复杂的消息交换,应该使用多个流。

对于即时的数据交换或通信,流的生命周期可能很短;如果需要,流的生命周期可能很长。

\subsection{流头}

所有libp2p流的一个Swarm特定要求是在两个对等体之间的每个流初始化时交换Header protobuf消息。此消息封装了在交换任何特定于流的数据或消息之前需要交换的流作用域信息。标头是键值对的序列,其中键是任意字符串,值是不强加任何特定编码的字节数组。每个键都可以对它所关联的数据使用适当的编码。

\begin{definition}[Header message]\label{def:headers-message}

\begin{lstlisting}[language=protobuf3]
syntax = "proto3";

package pb;

message Headers {
    repeated Header headers = 1;
}

message Header {
  string key = 1;
  bytes value = 2;
}
\end{lstlisting}
\end{definition}

在每个流初始化时,创建它的对等端都发送报头消息,不管它是否包含报头值。接收节点必须读取此消息并使用使用相同消息类型的响应头进行响应。这使得报头交换序列完成,任何其他流数据都可以根据协议传输。

标准头键名定义在这里:

\begin{enumerate}
\item tracing-span-context
\end{enumerate}


\subsection{用于跟踪\statusgreen的上下文的封装}

使用流头来交换P2P流作用域的跟踪跨度上下文。头键" trace -span-context"保留用于二进制编码的跟踪span上下文数据。该上下文应该用于跟踪消息。流启动器节点应该向响应节点提供跟踪跨上下文。该上下文是可选的,所有节点的功能必须与其他节点提供的span上下文相同,无论该节点是否配置了跟踪。


% \section{Swarm protocol basics\statusgreen}\label{spec:protocol:basics}
\section{Bzz handshake protocol \statusgreen}\label{spec:protocol:bzz}


bzz握手协议是在两个对等体建立连接之后以及在任何其他协议建立之前运行的协议。它传递有关对等体地址、网络ID和轻节点能力的信息。

握手协议只定义了一个流和三个消息:

\begin{definition}[Bzz handshake protocol  messages]\label{def:bzz-messages}

\begin{lstlisting}
// ID: /swarm/handshake/1.0.0/handshake

syntax = "proto3";

package handshake;

message Syn {
    bytes ObservedUnderlay = 1;
}

message Ack {
    BzzAddress Address = 1;
    uint64 NetworkID = 2;
    bool Light = 3;
}

message SynAck {
    Syn Syn = 1;
    Ack Ack = 2;
}

message BzzAddress {
    bytes Underlay = 1;
    bytes Signature = 2;
    bytes Overlay = 3;
}
\end{lstlisting}
\end{definition}

这个消息序列是受TCP三次握手的启发,以确保消息的可传递性。

在连接时,请求的对等体构造一个新的握手流,并发送一个\lstinline{Syn}消息,该消息带有远程观察到的与之进行握手的对等体的底层地址。之后,它等待来自响应对等体的\lstinline{SynAck}响应消息。在\lstinline{SynAck}消息中,响应器也发送它自己的\lstinline{Syn}消息,以及确认消息,确认消息应该包括它正确签名的BzzAddress。接收到的观测地址可以用来与本地已知的地址进行比较,以便在确认中发送更好的可广告地址(Underlay)。请求端从响应端接收到\lstinline{SynAck}消息,并确认接收到的\lstinline{Ack}信息是正确的后,向响应端发送一条\lstinline{Ack}消息作为确认。流在收到\lstinline{Ack}消息后被响应的对等体关闭。

如果网络id不匹配,或者没有遵循消息的确切顺序,则必须终止连接。

验证bzz地址,提取Overlay层、底层和签名。
Light是一个布尔字段,指示节点是否作为一个Light(而不是完整)节点运行。

握手后,每个对等体需要记住对方的以下数据:

\begin{itemize}
    \item Overlay地址-用于转发(参见\ref{spec:strategy:forwarding}),
    \item 底层地址——用于拨号,当连接驱动程序需要连接到对等体时传递给底层网络协议(参见\ref{spec:strategy:connection}),
    \item bzz地址签名——hive协议需要将该节点的信息传递给其他对等体(参见\ref{spec:protocol:hive}),
    \item 对端是否是轻节点。
\end{itemize}

% \subsection{Encapsulation of price information \statusred}



\section{Hive discovery  \statusgreen}\label{spec:protocol:hive}
\gloss{hive protocol}使节点能够交换与它们相关的其他对等点的信息,以便引导它们的连接(请参阅\ref{sec:bootstrapping})。通信的信息是已知远程对等点的Overlay地址和底层地址(参见\ref{spec:format:bzzaddress})。Overlay地址用于选择对等点,以实现所需的网络拓扑所需的连接模式。通过拨号选定的对等体来建立对等体连接需要底层地址。

\subsection{流和消息\statusgreen}


协议指定一个流有两个消息:

\begin{definition}[Hive protocol messages]\label{def:hive-messages}

\begin{lstlisting}
// /swarm/hive/1.0.0/peers
syntax = "proto3";

package hive;

message Peers {
    repeated BzzAddress peers = 1;
}

message BzzAddress {
    bytes Underlay = 1;
    bytes Signature = 2;
    bytes Overlay = 3;
}

\end{lstlisting}
\end{definition}

在连接的生命周期内,节点可以广播新接收到的对等体。这是通过通过\\\lstinline{/swarm/hive/1.0.0/peers}流发送\lstinline{Peers}消息来实现的。

在接收到对等点消息后,节点将在其\gloss{address book}中存储对等点信息,即包含该节点已知的对等点信息的数据结构。地址簿用于根据连接策略(\ref{spec:strategy:connection})向连接管理器推荐对等点,以引导kademlia拓扑(\ref{sec:kademlia-connectivity})。地址簿意味着跨会话持久化。

\subsubsection{发送端}

创建一个具有适当id的流,并通过流发送\lstinline{Peers}消息。这条消息没有回应。发送节点应该等待接收端关闭自己的流,然后再关闭流并继续前进。

\subsubsection{接收端}

When the stream is created, receiving node should wait for a \lstinline{Peers} message. After receiving the message, node should close its side of the stream to let the sender node know that the message was received, and move on with processing. If the new node was not known, it should also be forwarded to all connected peers closer to peer address then the node themselves.

\section{Retrieval  \statusorange}\label{spec:protocol:retrieval}
\input{specs/protocol/retrieval.tex}

\section{Push-syncing  \statusorange}\label{spec:protocol:push-sync}
\input{specs/protocol/push-sync.tex}

\section{Pull-syncing \statusorange}\label{spec:protocol:pull-sync}
拉同步的目的

邻居syncronisation

最大限度地利用资源战略



\section{SWAP settlement protocol \statusorange}\label{spec:protocol:swap}


\chapter{Strategies \statusorange}\label{spec:strategy}


The strategies clients follow will have a fundamental effect on the behaviour of the network and if they end up going against the preconceived design, the project may very well fail to suit user expectations. Such scenarios can easily turn fatal to the project, which is why it is instructive to err on the side of caution when change (or initially suggest) strategies for node behaviour.
The scope of strategies are defined as those aspects of the intended 'protocol' which cannot be directly observed or easily verified. As an example contrast the very act of forwarding an incoming retrieve request (strategy) with the act of using correctly formatted and serialised messages when doing so (protocol constraint). The latter can be immediately detected and be responded to by disconnecting and blacklisting offenders. In contrast, whether a node does forwarding in a way conducive to the desired network outcome is subtle to detect. 

Since we choose to work with the narrowest possible assumption of profit-maximising node operators, the choice of strategies ought to be course grained enough to evaluate the consequences of the options. In particular we must not allow for scenarios when even vaguely rational deviations from the recommended strategy have catastrophic effects.

In general incentives should be in place to guarantee that behaviour that is detrimental to the service incurs a risk of subsequent loss, deterrent upfront cost or opens up reciprocal  vulnerability. 

The constraints put on strategic choice are crucial in terms of rendering the game theory feasible to simulations and experiments or express them with simple enough analytical models to aid reasoning. 

\section{Connection  \statusorange}\label{spec:strategy:connection}

从两个流中的任何一个接收到的\lstinline{Peers}消息中发现的所有新的(以前不知道的)对等体以及新连接的拨号的对等体应该自动广播到比节点的基本覆盖地址更接近它们的所有连接的对等体。在形式上, 
节点$s$(发送方)通知现有的对等体$r$(接收方)关于对等体$p$,如果$\mathit{PO}(s, r) = \mathit{PO}(s, p)$。 

\subsection{连通性及其约束}

在没有资源约束的情况下,一个合适的存储节点的最佳连接是完全连接,即当一个节点将网络中的所有其他节点作为其对等节点时。支票簿契约也倾向于维护少量的对等连接,为保持连接或网络套接字短缺而产生的其他网络开销也暗示了这一点 
超过一定的网络规模,完全连接是禁止的。

如果使所有keepalive连接的吞吐量最大化,则相对增益也会最大化。如果网络竞争使一个节点不能立即转发给另一个对等体,则必须选择另一个对等体。 
假设网络中的一致性 
%(see appendix \ref{sec:distribution}) 
在对等连接数量有限的上下文中,吞吐量最大化可以通过以下连接模式实现:每个kademlia bin具有$2^b$的恒定基数,并且bin中的对等节点是平衡的,即匹配长度为$b$的每个不同的位前缀。
因此,互联互通策略可制定为:







\subsection{轻节点连接策略}

在连接拓扑的上下文中,轻节点是没有kademlia连接的节点。在极端情况下,轻节点应该能够通过单个连接逃脱。如果节点的对等节点没有完全连接,则不会向它们发送检索请求或推送同步请求。注意,如果一个节点错误地指示了它的状态,应该会造成最小的破坏。 

一个lightnode想要识别它的邻居,以便它连接到至少一个最接近它的$R$存储节点(其中$R$是确定最小邻居大小的冗余参数)。





\section{Forwarding  \statusorange}\label{spec:strategy:forwarding}
转发策略是指节点在转发Kademlia中对转发的消息和响应所遵循的与块检索和上传相关的处理过程。

注意,对于检索和推同步,协议并没有指定只允许转发检索请求和块传递。
为了获得稳定性,节点是否转发传入消息以及将消息转发到何处,应该由激励机制引导。

如\ref{sec:pricing}所述,节点分别制定并公布每个PO仓的价格。 
最优定价策略将反映PO箱,因为更接近的箱的价格将单调下降。 

有足够的 
    
 
\subsubsection{重试}
%disconnection

推同步和检索消息都使用了反向保护(即传递回响应),转发补偿仅在响应发送后才被计算在内。这将激励节点监视发送下游消息的对等连接。特别是,如果下游对等体在收到响应之前断开连接,则应该重复转发并在另一个对等体上尝试。


%price-driven strategy

% acyclicity


 %\subsubsection{}

% multiple requests




%peer selection and pricing
\section{Pricing  \statusorange}\label{spec:strategy:pricing}
下面我们将描述3种策略:每一种都比前一种更复杂。

被动:弱卡特尔-硬有线固定价格表
最小响应性:如果没有保证边界,则不对对等体进行转发。 

反应性-对下游价格上涨的反应,探索其他同行。如果没有其他选择,就会提高价格。如果下游的同行降低价格,它也会效仿。

积极主动:回复 


\subsection{被动策略\statusorange}\label{sec:pricing-retrieval}
下面的策略是节点如何使用上面描述的协议的基本方法。该策略的意图是尽可能容易地实施,而balbalbala

\subsubsection{保证金}
“ChunkDeliveryRequest”的节点价格总是至少是他想要的“保证金”+最便宜的上游peer的价格。 

\subsubsection{填充priceInformationRegistry}
最初,节点只知道从最接近的容器中提供内容的价格(price == ' margin ')。当这个节点收到他的第一个请求,提供的内容是距离他最接近的存储库边界一个接近订单的内容时,他首先会检查这个请求的价格是否超过他的边际。如果是这种情况,他将以0的价格转发' retriverequest '(因为状态还没有初始化)。如果在适当的仓中的上游对等体没有改变他的价格,这个对等体将返回一个' NewPrice '消息,其中价格等于下游对等体的' margin '。如果原始价格至少是“保证金”的2倍,节点可以以“保证金”的价格转发请求(并将另一个马林自己收进口袋)。从这一点开始,“状态”将为一个对等体和一个邻近体初始化。随后发送给这个peer的' retriverequests '将以' margin '的' price '发送。更远的区块价格更高。通过这个迭代过程,同伴将很快了解到这一点。如果所有对等点应用相同的' margin ', ' ChunkDelivery '的价格等于请求者和块之间的接近顺序乘以' margin '。如果对等的请求一块价格小于他的“保证金”+上游同行最便宜的价格,本,请求不转发,但立即回答“NewPrice”的信息与价格等于“保证金”+上游同行最便宜的价格,本。

\subsubsection{对价格差异的反应}
网络中没有对等收取“保证金”不同于默认的边缘,只“NewPrice”消息将发送到同龄人没有完全初始化他们的“状态”,同等的距离所请求的块和同行之间在一个特定的箱子,一个节点没有一块基于价格的首选供应商。但是,如果网络中的任何一个peer改变了他的价格,基于价格的偏好就会出现:

对于任何块请求:
—选择要转发请求的bin
—选择PO距离chunk最近的对等体
—选择价格最低的对等体
-如果有多个对等点具有相同的价格,则执行轮询负载均衡

由此,我们可以得出结论,如果对等体降低价格,他将收到更多的‘retriverequests’;如果对等体提高价格,他将收到更少的‘retriverequests’。与价格较低的上游peer连接的节点可能会决定在此距离上降低价格,以便接收更多请求。

\subsubsection{通知价格变动}
这是网络的利益,当有价格变化时,这将尽快传播到相关各方。价格变化可以由“保证金”的变化引起,或者由上游对等方的价格变化引起。在这两种情况下,对等体可能决定不通知其他对等体价格的下降——惟一的变化是它将以比以前更低的价格接受' retriverequests '。然而,如果不通知下游对等体,节点可能无法直接获得预期数量的额外客户端,因为下游对等体可能永远不会提出比以前更低的价格。出于这个原因,我们建议节点主动发送“NewPrice”消息给他的同伴,本质上是要求他们更新他们的“priceInformationRegistry”。在主动' NewPrice '消息的情况下,' chunkReference '可能是合成的,这意味着它不必响应一个实际的块;重要的是同行们更新他们的“价格信息注册表”。

\textbf{Overpriced bins}
自一个节点的价格取决于他的上游同行的价格,它会发生改变,他所有的上游同学之间的关系在某本是由昂贵的peers-effectively填充节点本身过于昂贵和下游的同行。为了解决这个问题,我们建议进行统计分析:如果一个节点在他的所有箱子上都有合理的价格,那么他将收到所有箱子相同数量的请求。如果一个节点接收到的对某个箱子的请求比其他箱子的请求少得多,他就会将那些被认为价格过高的对等体标记为无功能,蜂巢协议就会启动,建议新的对等体连接。 

\textbf{Reacting to demand increases/decreases}

如果一个节点的价格降低到最便宜,那么它将获得更多的流量。如果不能对流量的增加作出适当的响应,节点可能被迫在D时脱机

\subsubsection{配置选项}
将添加以下配置选项。 

\begin{verbatim}

|名称|单位|默认|
| -------- | -------- | -------- |
'保证金'支票簿的基础货币
| ' period ' | duration in seconds | 300 |
| ' maximum_upstream_bandwidth ' |带宽使用,以字节为单位每'周期' | TODO |
| ' high_water_mark ' | percentage:(upstream bandwidth used / ' period ') / (' maximum_upstream_bandwidth ') | 80%     |
| ' low_water_mark ' | percentage:(upstream bandwidth used / ' period ') / (' maximum_upstream_bandwidth ') | 20%     |
| ' margin_change ' | percentage | 1%     |

\end{verbatim}

下面是如何使用这些变量的。

\subsection{节点fabbd的行为}
1)给定一个特定的\lstinline{`chunkReference`},节点选择将$`ChunkRequestMessage`$发送给该bin中所有节点中价格最低的节点。
2)如果从一个' chunkRequestMessage '返回一个' newPrice '消息,节点更新' priceInformationRegistry '并再次发送' chunkRequestMessage ',并根据更新的' priceInformationRegistry '选择对等体。 
4)当对等端请求一个特定的ChunkRequestMessage,并且在其本地存储中没有' chunkContent ',它转发' ChunkRequestMessage '(见1),价格等于它收到的' ChunkRequestMessage '减去他的' margin '。
一个节点跟踪他的“带宽\_usage”(\lstinline{(upstream bandwidth used / `period`) / (`maximum\_upstream\_bandwidth`)})
6)当' bandWidth\_usage ' > ' high\_water\_mark '时,' margin '将为(1+ ' margin\_change ') * ' margin '
7)当' bandWidth\_usage ' < ' low\_water\_mark '时,' margin '将是(1- ' margin\_change ') * ' margin '。“保证金\_change”表示 
8)当一个节点可以得到更便宜的价格而不是“块”要求由“ChunkRequestMessage”(上游同行改变价格,数量的啤酒花小于平均请求相同的距离),他回答“ChunkDeliveryMessage”,将基于“ChunkRequestMessage”定价。 
9)当一个节点期望能够提供比他的同行目前知道的长期更低的价格时,他发送一个“NewPrice”消息给他的同行。 


\section{Accounting and settlement  \statusorange}\label{spec:strategy:swap}

付款阈值

断开阈值

收到支票

缓存策略

支票簿平衡 

topup策略



\section{Push-syncing  \statusorange}\label{spec:strategy:push-sync}
\input{specs/strategy/push-sync.tex}

\section{Pull-syncing  \statusorange}\label{spec:strategy:pull-sync}
拉同步的目的

邻居syncronisation

最大限度地利用资源战略



\section{Garbage collection \statusorange}\label{spec:strategy:garbage-collection}
垃圾收集是清除节点的本地块存储过程的美化名称。 
这里的假设是,存储是稀缺资源,随着时间的推移,节点将经历容量短缺,并面临决定删除哪些现有的块,以便为新到达的块腾出空间的问题。 

垃圾收集策略必须反映块的经济潜力,也就是说,它应该使节点在存储补偿上获得的利润最大化。从节点采用特定策略的假设中推导出Swarm作为网络的高级行为,只有当该策略与激励设计最佳匹配时才有效。

如果只有Swap是可操作的,那么一个非常简单的垃圾收集策略就足够了。每当一个块被检索时,服务它的节点就会获得SWAP收入。因此,理想情况下,那些经常检索的块应该优先于那些很少检索的块。

访问次数的最好预测器是过去的访问次数。对于存在和检索的块,这些模型工作得很好,但它们只适合区分流行的内容,而不是很少访问的内容。新区块的先验估计可以依赖于邮资价值,它直接映射到保证的收入。  


可用,取值一致的存储

按块平衡进行索引

更新邮票价值 

结合交换利润 

让我们按照最近的顺序用负整数索引纪元,例如,当前纪元是0,最近的$-1$,前一个$-2$,等等。 
让$n$为过去时代的截止内存大小。
让$\mathit{Hits}(e)$表示在epoch $e$期间为块提供的命中次数。 

让我们根据过去的观察建立一个流行预测模型,使用预测能力随时间呈指数衰减的假设。 

\begin{equation}
    \mathit{Hits}(0) \defeq \frac{\sum_{e=0}^n \frac{\mathit{Hits}(e)}{c^e}}{c^{n-1}}
\end{equation}

\chapter{API-s}\label{spec:api}


\begin{definition}[HTTP status codes used in swarm]
\begin{tabular}{l|p{0.25\linewidth}|p{0.6\linewidth}}
103 & Checkpoint & returns temporary root hash for resumable uploads
\\\hline
200 & ok &
\\
201 & Created & returned by POST requests upon successful creation of file/manifest/tag/stamp/
% \\
% 206 & Partial Content &
\\\hline
400 & Bad request & returned if request or its parameters are not well formed or missing
\\
401 & Unauthorized & returned by access control if auth fails
\\
402$^{*}$ & Payment Required & returned if no swap balance or missing/invalid postage stamp
\\
403 & Forbidden & returned if retrieved chunk is encrypted  but the reference has no decryption key
\\     
404 & Not Found &
returned if a local prerequisite is not found or manifest path does not exist.
\\
405$^{*}$& Method Not Allowed & HTTP verb  not allowed for this endpoint
\\
406$^{*}$
%
& Not Acceptable & No format acceptable by the {ACCEPT} header explicit in the request. 
\\
408 & Request Timeout & Retrieve requests fallback error after TTL passed
\\
% 411 & Length Required & returned by chunk upload API if length of file uploaded is beyond limit
% 412 & Precondition Failed (RFC 7232)
% \\
413 & Payload Too Large  &
Payload size exceeds maximum chunk size or span given (chunk API)
\\
414 & URI Too Long  & manifest path  $>32 $
\\
416 & Range Not Satisfiable  & offset in range query out of range
\\
420 & Enhance your calm & returned when recovery was initiated but retrieval timed out
\\
422 & Unprocessable Entity & returned by the blockchain external API if eth api returns an error or by single owner chunk post API if owner/id pair already exists
% 417 & Expectation Failed
% 421 & Misdirected Request (RFC 7540)
% 451 & Unavailable For Legal Reasons (RFC 7725)
\end{tabular}

\footnotesize{$^{*}$Generic errors detectable before endpoint API call so not documented.}
\end{definition}


\section{External API requirements\statusorange}\label{spec:api:external}

\subsection{签名者}\label{spec:api:signer}
% \input{specs/api/signer.tex}

% \UseRawInputEncoding
\begin{apiRoute}{GET}{/sign/\param{id}/\param{document}}{ECDSA signature}{
}{ }
\begin{routeParameter} \routeParamItem{id}{hex string - eth address}
\routeParamItem{document}{(hex) string -  document to sign (is prefixed and hashed before signing)}
\end{routeParameter} \begin{routeResponse}{application/json}
\responseItem{200}{ok}{}
\responseItem{401}{Unauthorised}{failed authentication on existing identity} 
\responseItem{404}{Not found}{unknown identity} 
\end{routeResponse} 
\end{apiRoute}

\begin{apiRoute}{GET}{/dh/\param{id}/\param{pubkey}}{Diffie-Hellman shared secret}{
}{ }
\begin{routeParameter} \routeParamItem{id}{hex string - eth address}
\routeParamItem{pubkey}{hex string - represents the remote party in the shared secret arrangement}
\end{routeParameter} \begin{routeResponse}{application/json}
\responseItem{200}{ok}{} \responseItem{401}{Unauthorised}{failed authentication on existing identity}
\responseItem{404}{Not found}{unknown identity}
\end{routeResponse} \end{apiRoute}


\subsection{区块链\statusgreen}\label{spec:api:blockchain}
% \input{specs/api/blockchain.tex}

\begin{apiRoute}{GET}{/eth/\param{contract}/\param{function}/\param{args}}{ethereum API call}{
}{ }
\begin{routeParameter}
\routeParamItem{contract}{hex string, eth address of contract}
\routeParamItem{function}{endpoint within contract}
\routeParamItem{args}{arguments for the eth API call}
\end{routeParameter} \begin{routeResponse}{application/json}
\responseItem{200}{ok}{}  \responseItem{400}{Bad request}{unknown contract or function endpoint given} 
\responseItem{404}{Not Found}{unknown contract or function endpoint} \responseItem{422}{Unprocessable entity}{incorrect ABI,  error by eth API}
\end{routeResponse} \end{apiRoute}

\begin{apiRoute}{POST}{/eth/\param{contract}/\param{function}/\param{args}}{ethereum API send transaction}{
}{ }
\begin{routeParameter}
\routeParamItem{contract}{hex string, eth address of contract}
\routeParamItem{function}{endpoint within contract}
\routeParamItem{args}{arguments for the eth API call}
\end{routeParameter} \begin{routeResponse}{application/json}
\responseItem{200}{ok}{}
\responseItem{401}{Unauthorised}{failure signing transaction}
\responseItem{400}{Bad request}{unknown contract or function endpoint given} 
\responseItem{404}{Not Found}{unknown contract or function endpoint} 
\responseItem{422}{Unprocessable entity}{incorrect ABI,  error by eth API}
\end{routeResponse} \end{apiRoute}


\subsection{用户输入}\label{spec:api:input}


\begin{apiRoute}{GET}{/input/\param{id}}{
}{ }
\begin{routeParameter}
\routeParamItem{id}{hex string - eth address of the persona the input is expected from}
\routeParamItem{note}{the question to be answered or instruction to select}
\end{routeParameter} \begin{routeResponse}{application/json}
\responseItem{200}{ok}{}
\end{routeResponse} \end{apiRoute}



\section{Storage API \statusyellow}\label{spec:api:storage}
\subsection{块}

\begin{apiRoute}{GET}{/chunk/\{reference\}}{Retrieve chunk as in \ref{def:retrieve}}{
}{ }

\begin{routeParameter} 
\routeParamItem{reference}{hex string}
\end{routeParameter}
\begin{routeResponse}{application/json}
\responseItem{200}{ok}{Chunk data as response body}
\responseItem{403}{Forbidden}{chunk encrypted but no decryption key in reference}
\responseItem{408}{Request Timeout}{} 
\responseItem{420}{Enhance your calm}{Recovery initiated but request timed out} 
\end{routeResponse}
\end{apiRoute}




\begin{apiRoute}{POST}{/chunk/(?span=\{span\})}{Create chunk as in \ref{def:store}}{
}{ }

\begin{queryParameter} 
\queryParamItem{span}{integer}
\end{queryParameter}

\begin{headerParameter} 
\headerParamItem{SWARM-TAG}{hex string}
\headerParamItem{SWARM-STAMP}{hex string}
\headerParamItem{SWARM-ENCRYPTION}{hex string}
\headerParamItem{SWARM-PIN}{bool}
\headerParamItem{SWARM-PARITIES}{integer}
\end{headerParameter}

\begin{routeResponse}{application/json}
\responseItem{200}{ok}{Minimal manifest entry in response body}
\responseItem{400}{Bad request}{Span parameter not well formed. } 
\responseItem{413}{Payload Too Large}{Payload size exceeds span value or 4096} 

\end{routeResponse}
\end{apiRoute}




\begin{apiRoute}{GET}{/soc/\param{owner}/\param{id}}{Retrieve single owner chunk as in \ref{def:soc-retrieve} }{
}{ }

\begin{routeParameter} 
\routeParamItem{owner}{eth address of single owner}
\routeParamItem{id}{identifier within owner namespace}
\end{routeParameter}
\begin{queryParameter} 
\queryParamItem{key}{string}
\end{queryParameter}
\begin{routeResponse}{application/json}
\responseItem{200}{ok}{single owner chunk payload in response body}
\responseItem{400}{Bad request}{if owner, id or key is not well formed}
\responseItem{403}{Forbidden}{Single owner chunk encrypted but no decryption key given}
\responseItem{408}{Request Timeout}{} 
\responseItem{420}{Enhance your calm}{Recovery initiated but request timed out}
\end{routeResponse}
\end{apiRoute}




\begin{apiRoute}{POST}{/soc/\param{owner}/\param{id}?span=\param{span}}{Create new single owner chunk as in \ref{def:soc-store} }{
}{ }

\begin{routeParameter} 
\routeParamItem{owner}{eth address of single owner}
\routeParamItem{id}{identifier within owner namespace}
\end{routeParameter}
\begin{headerParameter} 
\headerParamItem{SWARM-TAG}{hex string}
\headerParamItem{SWARM-STAMP}{hex string}
\headerParamItem{SWARM-ENCRYPTION}{hex string}
\headerParamItem{SWARM-PIN}{bool}
\headerParamItem{SWARM-PARITIES}{integer}
\end{headerParameter}
\begin{queryParameter} 
\queryParamItem{span}{integer}
\end{queryParameter}
\begin{routeResponse}{application/json}
\responseItem{201}{Created}{Reference in response body}
\responseItem{400}{Bad request}{owner,  id or span is not well formed} 
\responseItem{401}{Unauthorized}{signing fails} 
\responseItem{404}{Not found}{owner keypair is found} 
\responseItem{413}{Payload Too Large}{Payload size exceeds span value or 4096} 
\responseItem{422}{Unprocessable entity}{owner/id pair already exists} 
\end{routeResponse}
\end{apiRoute}

\subsection{文件\statusgreen}\label{spec:api:file}

\begin{apiRoute}{POST}{/file/}{Path description as in \ref{def:swarm-hash}}{
}{ }

\begin{headerParameter} 
\headerParamItem{SWARM-TAG}{hex string}
\headerParamItem{SWARM-STAMP}{hex string}
\headerParamItem{SWARM-ENCRYPTION}{hex string}
\headerParamItem{SWARM-PIN}{bool}
\headerParamItem{SWARM-PARITIES}{integer}
\end{headerParameter}
\begin{routeResponse}{application/json}
\responseItem{201}{Created}{Minimal manifest entry as response body}


\end{routeResponse}
\end{apiRoute}






\begin{apiRoute}{PUT}{/file/\param{reference} }{Append body to file, returns new reference. Note that any intermediate chunk of a file is a file.}{
}{ }

\begin{routeParameter} 
\routeParamItem{reference}{hex string}
\routeParamItem{file}{binary as request body}
\end{routeParameter}
\begin{headerParameter} 
\headerParamItem{SWARM-TAG}{hex string}
\headerParamItem{SWARM-STAMP}{hex string}
\headerParamItem{SWARM-ENCRYPTION}{hex string}
\headerParamItem{SWARM-PIN}{bool}
\headerParamItem{SWARM-PARITIES}{integer}
\end{headerParameter}
\begin{routeResponse}{application/json}
\responseItem{201}{Created}{Minimal manifest entry as request body}
\responseItem{400}{Bad Request}{Reference is not well formed}
\responseItem{403}{Forbidden}{Encrypted content but no decryption key in. reference}
\responseItem{408}{Request Timeout}{Retrieval of file to append to times out} 
\responseItem{420}{Enhance your calm}{Recovery initiated but request timed out}
\responseItem{413}{Payload Too Large}{}
\end{routeResponse}
\end{apiRoute}



\begin{apiRoute}{GET}{/file/\param{reference}}{Retrieve file by reference as in \ref{def:file-retrieval}}{
}{ }

\begin{routeParameter} 
\routeParamItem{reference}{string}
\end{routeParameter}
\begin{routeResponse}{application/json}
\responseItem{200}{ok}{file contents streamed in response body}
\responseItem{400}{Bad Request}{Reference is not well formed}
\responseItem{403}{Forbidden}{Encrypted content but no decryption key in reference}
\responseItem{408}{Request Timeout}{} 
\responseItem{416}{Range Not Satisfiable}{Offset in range query out of range}
\responseItem{420}{Enhance your calm}{Recovery initiated but request timed out}
\end{routeResponse}
\end{apiRoute}


\subsection{清单\statusgreen}\label{spec:api:manifest}




\begin{apiRoute}{GET}{/manifest/\param{reference}/\param{path}}{Lookup entry by path in manifest, see \ref{def:manifests-lookup}}{
}{ }

\begin{routeParameter} 
\routeParamItem{reference}{hex string}
\routeParamItem{path}{string}
\end{routeParameter}
\begin{routeResponse}{application/json}
\responseItem{200}{ok}{Manifest entry in response body}
\responseItem{400}{Bad Request}{Reference or path is not well formed}
\responseItem{403}{Forbidden}{Encrypted content but no decryption key in reference}
\responseItem{404}{Not found}{Path does not exist}
\responseItem{408}{Request Timeout}{Timeout retrieving referenced manifest}
\responseItem{420}{Enhance your calm}{Recovery initiated but request timed out}
\end{routeResponse}
\end{apiRoute}



\begin{apiRoute}{DELETE}{/manifest/\param{reference}/\param{path} }{Delete entry on path in referenced manifest, see \ref{def:manifest-remove}}{
}{ }

\begin{routeParameter} 
\routeParamItem{reference}{hex string}
\routeParamItem{path}{string}
\end{routeParameter}
\begin{routeResponse}{application/json}
\responseItem{204}{No content}{Successful deletion}
\responseItem{400}{Bad Request}{Reference or path is not well formed}
\responseItem{403}{Forbidden}{Encrypted content but no decryption key in reference}
% \responseItem{404}{Not found}{Path does not exist}
\responseItem{408}{Request Timeout}{Timeout retrieving referenced manifest}
\responseItem{420}{Enhance your calm}{Recovery initiated but request timed out}
% \responseItem{404}{Not found}{Path does not exist}
\end{routeResponse}
\end{apiRoute}



\begin{apiRoute}{PUT}{/manifest/\param{reference}/\param{path}}{Update manifest as in \ref{def:manifest-update}}{
}{ }

\begin{routeParameter} 
\routeParamItem{reference}{hex string}
\routeParamItem{path}{(hex) string}
\end{routeParameter}
\begin{routeResponse}{application/json}
\responseItem{200}{ok}{}
\responseItem{400}{Bad Request}{Reference is not well formed}
\responseItem{403}{Forbidden}{Encrypted content but no decryption key in reference}
\responseItem{404}{Not found}{Path does not exist}
\responseItem{408}{Request Timeout}{Timeout retrieving referenced manifest}
\responseItem{414}{URI Too Long}{Path exceeds 32 byte limit}
\responseItem{420}{Enhance your calm}{Recovery initiated but request timed out}

\end{routeResponse}
\end{apiRoute}




\begin{apiRoute}{POST}{/manifest/\param{old}/\param{new}}{Merge manifests as in \ref{def:manifest-merge}}{
}{ }

\begin{routeParameter} 
\routeParamItem{old}{hex string}
\routeParamItem{new}{hex string}
\end{routeParameter}
\begin{routeResponse}{application/json}
\responseItem{201}{Created}{Reference in response body}
\responseItem{400}{Bad Request}{Reference is not well formed}
\responseItem{403}{Forbidden}{Encrypted content but no decryption key in reference}
\responseItem{404}{Not found}{Path does not exist}
\responseItem{408}{Request Timeout}{Timeout retrieving referenced manifest}
\responseItem{414}{URI Too Long}{Path exceeds 32 byte limit}
\responseItem{420}{Enhance your calm}{Recovery initiated but request timed out}
\end{routeResponse}
\end{apiRoute}

\subsection{高级存储API \statusyellow}\label{spec:api:manifest}


\begin{apiRoute}{GET}{/bzz:/\param{host}/\param{path}}{Download file}{
}{ }

\begin{routeParameter} 
\routeParamItem{host}{}
\routeParamItem{path}{}
\end{routeParameter}
\begin{headerParameter} 
\headerParamItem{SWARM-TAG}{hex string}
\headerParamItem{SWARM-STAMP}{hex string}
\headerParamItem{SWARM-ENCRYPTION}{hex string}
\headerParamItem{SWARM-PIN}{bool}
\headerParamItem{SWARM-PARITIES}{integer}
\end{headerParameter}
\begin{routeResponse}{application/json}
\responseItem{200}{ok}{}
\responseItem{400}{Bad Request}{Host/Reference or path is not well formed}
\responseItem{401}{Unauthorized}{Access denied: AC unlock failed}
\responseItem{403}{Forbidden}{Encrypted content but no decryption key in reference}
\responseItem{404}{Not found}{Host cannot be resolved or Path does not exist}
\responseItem{408}{Request Timeout}{Timeout retrieving referenced manifest}
\responseItem{420}{Enhance your calm}{Recovery initiated but request timed out}
\end{routeResponse}
\end{apiRoute}



\begin{apiRoute}{PUT}{/bzz:/\param{host}/\param{path}}{Append upload to file referenced, add new entry to path, returns new manifest}{
}{ }

\begin{routeParameter} 
\routeParamItem{file/collection}{as request body}
\end{routeParameter}
\begin{headerParameter} 
\headerParamItem{SWARM-TAG}{hex string}
\headerParamItem{SWARM-STAMP}{hex string}
\headerParamItem{SWARM-ENCRYPTION}{hex string}
\headerParamItem{SWARM-PIN}{bool}
\headerParamItem{SWARM-PARITIES}{integer}
\end{headerParameter}
\begin{routeResponse}{application/json}
\responseItem{201}{Created}{New manifest root reference in response body}
\responseItem{400}{Bad Request}{Host/Reference is not well formed}
\responseItem{401}{Unauthorized}{Accessdenied: AC unlock failed}
\responseItem{403}{Forbidden}{Encrypted content but no decryption key in reference}
\responseItem{404}{Not found}{Host cannot be resolved or Path does not exist}
\responseItem{408}{Request Timeout}{Timeout retrieving referenced manifest}
\responseItem{414}{URI Too Long}{Path exceeds 32 byte limit}
\responseItem{420}{Enhance your calm}{Recovery initiated but request timed out}
\end{routeResponse}
\end{apiRoute}




\begin{apiRoute}{POST}{/bzz:/\param{host}/\param{path}}{Upload file or collection, returns new manifest root reference}{
}{ }

\begin{routeParameter} 
\routeParamItem{file/collection}{as request body}
\end{routeParameter}
\begin{headerParameter} 
\headerParamItem{SWARM-TAG}{hex string}
\headerParamItem{SWARM-STAMP}{hex string}
\headerParamItem{SWARM-ENCRYPTION}{hex string}
\headerParamItem{SWARM-PIN}{bool}
\headerParamItem{SWARM-PARITIES}{integer}
\end{headerParameter}
\begin{routeResponse}{application/json}
\responseItem{201}{Created}{New manifest root reference in response body}
\responseItem{400}{Bad Request}{Host/Reference is not well formed}
\responseItem{401}{Unauthorized}{Accessdenied: AC unlock failed}
\responseItem{403}{Forbidden}{Encrypted content but no decryption key in reference}
\responseItem{404}{Not found}{Host cannot be resolved or Path does not exist}
\responseItem{408}{Request Timeout}{Timeout retrieving referenced manifest}
\responseItem{414}{URI Too Long}{Path exceeds 32 byte limit}
\responseItem{420}{Enhance your calm}{Recovery initiated but request timed out}
\end{routeResponse}
\end{apiRoute}





\subsection{Tags \statusyellow}\label{spec:api:tags}

\begin{apiRoute}{POST}{/tags}{Create new tag}{
}{ }

\begin{routeResponse}{application/json}
\responseItem{201}{Created}{ID in response body}
\end{routeResponse}
\end{apiRoute}




\begin{apiRoute}{GET}{/tags?offset=\param{offset}\&length=\param{length}}{Get all tags}{
}{ }
\begin{queryParameter} 
\queryParamItem{offset}{integer}
\queryParamItem{length}{integer}
\end{queryParameter}

\begin{routeResponse}{application/json}
\responseItem{200}{ok}{}
\end{routeResponse}
\end{apiRoute}




\begin{apiRoute}{GET}{/tags/\param{id}}{View details tag with given ID}{
}{ }

\begin{routeParameter} 
\routeParamItem{id}{string}
\end{routeParameter}
\begin{routeResponse}{application/json}
\responseItem{200}{ok}{}
\responseItem{400}{Bad Request}{ID not well formed}
\responseItem{404}{Not found}{Tag with ID does not exists.}
\end{routeResponse}
\end{apiRoute}



\begin{apiRoute}{DELETE}{/tags/\param{id}}{Path description}{
}{ }

\begin{routeParameter} 
\routeParamItem{id}{string}
\end{routeParameter}
\begin{routeResponse}{application/json}
\responseItem{204}{No content}{}
\responseItem{400}{Bad Request}{ID not well formed}
\responseItem{404}{Not found}{Tag with ID does not exists.}

\end{routeResponse}
\end{apiRoute}



\subsection{Pinning
\statusyellow}\label{spec:api:pinning}

\begin{apiRoute}{GET}{/pin/?offset=\param{offset}\&length=\param{length}}{List Pinned Content and metadata}{
}{ }

\begin{queryParameter} 
\queryParamItem{offset}{integer}
\queryParamItem{length}{integer}
\end{queryParameter}
\begin{routeResponse}{application/json}
\responseItem{200}{ok}{}
\end{routeResponse}
\end{apiRoute}


\begin{apiRoute}{GET}{/pin/\param{id}}{View Pinned Content and metadata}{
}{ }

\begin{queryParameter} 
\queryParamItem{id}{hex string}{}
\end{queryParameter}
\begin{routeResponse}{application/json}
\responseItem{200}{ok}{pin in response body}
\responseItem{400}{Bad Request}{ID not well formed}
\responseItem{404}{Not found}{Pin with ID does not exists.}
\end{routeResponse}
\end{apiRoute}




\begin{apiRoute}{PUT}{/pin/\param{id}}{Pin an already uploaded content}{
}{ }

\begin{routeParameter} 
\routeParamItem{id}{hex string}
\end{routeParameter}
\begin{routeResponse}{application/json}
\responseItem{200}{ok}{}
\responseItem{400}{Bad Request}{ID not well formed}
\responseItem{404}{Not found}{Pin with ID does not exists.}
\end{routeResponse}
\end{apiRoute}




% \begin{apiRoute}{PUT}{/pin/\{id\}}{Path description?}
% {
% }
% { }

% \begin{routeParameter} 
% \routeParamItem{id}{}
% \end{routeParameter}
% \begin{routeResponse}{application/json}
% \begin{routeResponseItem}{200}{Ok}
% \begin{routeResponseItemBody}
  
% \end{routeResponseItemBody}
% \end{routeResponseItem}
% \end{routeResponse}
% \end{apiRoute}




\begin{apiRoute}{DELETE}{/pin/\param{id}}{Remove pinning from content}{
}{ }

\begin{routeParameter} 
\routeParamItem{id}{}
\end{routeParameter}
\begin{routeResponse}{application/json}
\responseItem{204}{No Content}{}
\responseItem{400}{Bad Request}{ID not well formed}
\responseItem{404}{Not found}{Pin with ID does not exists.}
\end{routeResponse}
\end{apiRoute}


% Content can be pinned in two different ways. One is during the content upload and the other is thereafter.

%     Pinning during upload

%     Add a header “x-swarm-pin” and set it to “true” when uploading content. This will upload the file and then pin it too. This method can be used for Tar, Multipart and raw file uploads too.

% curl -H "Content-Type: application/x-tar" -H "x-swarm-pin: true"  --data-binary @files.tar http://localhost:8500/bzz:/

%     Pinning after upload

%     If an already uploaded content needs to be pinned, the following HTTP API should be used.

% to pin a Swarm collection
% POST /bzz-pin:/<MANIFEST OR ENS NAME>

% to pin a RAW file in Swarm
% POST /bzz-pin:/<SWARM RAW FILE HASH>/?raw=true

% Note

% When pinning a already uploaded file, make sure that the entire file content is available locally by issuing a download once.
% Unpinning Content

% An already pinned file can be unpinned at anytime. Once the collection is unpinned, the contents will follow the FIFO rule and may be garbage collected in future.

% DELETE /bzz-pin:/<MANIFEST OR ENS NAME OR SWARM RAW FILE HASH>

% Listing Pinning Info

% Pinned contents and their information can be viewed at any point using this API. Information includes The pinned hash, whether the pinned content is a collection or RAW file, the pinned content size in bytes and the no of time the content is pinned.

% % \lstset{basicstyle=\footnotesize}
% % \lstinline{GET /bzz-pin:/}

% \begin{lstlisting}

% GET /bzz-pin:/

% [
%  { "Address"    : "0x94f78a45c7897957809544aa6d68aa7ad35df695713895953b885aca274bd955",
%   "IsRaw"      : "false",
%   "FileSize"   : "12046",
%   "PinCounter" : "2",
%  },
%  { "Address"    : "0xccef599d1a13bed9989e424011aed2c023fce25917864cd7de38a761567410b8",
%   "IsRaw"      : "true",
%   "FileSize"   : "146",
%   "PinCounter" : "5",
%  },
% ]

    
% \end{lstlisting}
% Note

% The information will be returned in JSON format shown above


\subsection{Swap and chequebook\statusorange}\label{spec:api:swap}

付款阈值

断开阈值

收到支票

缓存策略

支票簿平衡 

topup策略



\subsection{Postage stamps \statusorange}\label{spec:api:postage}

\begin{apiRoute}{GET}{/stamp?offset=\param{offset}\&length=\param{length}}{View all postage stamps}{
}{ }
\begin{queryParameter} 
\queryParamItem{offset}{integer}
\queryParamItem{length}{integer}
\end{queryParameter}

\begin{routeResponse}{application/json}
\responseItem{200}{ok}{}
\end{routeResponse}
\end{apiRoute}



\begin{apiRoute}{GET}{/stamp/\param{id}}{View postage stamp with id}{
}{ }

\begin{routeParameter} 
\routeParamItem{id}{}
\end{routeParameter}
\begin{routeResponse}{application/json}
\responseItem{200}{ok}{}
\responseItem{400}{Bad Request}{ID not well formed}
\responseItem{404}{Not found}{Stamp with ID does not exists.}
\end{routeResponse}
\end{apiRoute}




\begin{apiRoute}{PUT}{/stamp/\param{id}}{Top up postage stamp with id}{
}{ }

\begin{routeParameter} 
\routeParamItem{id}{}
\end{routeParameter}
\begin{queryParameter} 
\queryParamItem{amount}{integer}
\end{queryParameter}

\begin{routeResponse}{application/json}
\responseItem{200}{ok}{}
\responseItem{400}{Bad Request}{ID not well formed}
\responseItem{404}{Not found}{Stamp with ID does not exists.}
\end{routeResponse}
\end{apiRoute}



\begin{apiRoute}{DELETE}{/stamp/\param{id}}{Drain and expire stamp with id}{
}{ }

\begin{routeParameter} 
\routeParamItem{id}{}
\end{routeParameter}
\begin{routeResponse}{application/json}
\responseItem{200}{ok}{}
\responseItem{400}{Bad Request}{ID not well formed}
\responseItem{404}{Not found}{Stamp with ID does not exists.}
\end{routeResponse}
\end{apiRoute}


\begin{apiRoute}{POST}{/stamp/}{Create a new postage stamp, return ID in reponse body}{
}{ }

\begin{routeParameter} 
\end{routeParameter}
\begin{routeResponse}{application/json}
\responseItem{201}{created}{ID in response body}
\end{routeResponse}
\end{apiRoute}



\subsection{Access Control  \statusgreen}\label{spec:api:access-control}

\begin{apiRoute}{POST}{/access/\param{address} }{Lock ACT for address as in \ref{def:ac-api}}{
}{ }

\begin{routeParameter} 
\routeParamItem{address}{hex string}
\end{routeParameter}
\begin{routeResponse}{application/json}
\responseItem{201}{Created}{root access manifest reference in response body}
\responseItem{400}{Bad Request}{Encrypted content but no decryption key in reference}
\responseItem{403}{Forbidden}{Encrypted content but no decryption key in reference}
\responseItem{404}{Not found}{}
\responseItem{408}{Request Timeout}{Timeout retrieving referenced manifest}
\responseItem{420}{Enhance your calm}{Recovery initiated but request timed out}
\end{routeResponse}
\end{apiRoute}



\begin{apiRoute}{GET}{/access/\param{address} }{Unlock ACT for address as in \ref{def:ac-api}}{
}{ }

\begin{routeParameter} 
\routeParamItem{address}{hex string}
\end{routeParameter}
\begin{routeResponse}{application/json}
\responseItem{200}{ok}{}
\responseItem{400}{Bad Request}{Address not well formed}
\responseItem{401}{Unauthorized}{Access denied: AC unlock failed}
\responseItem{403}{Forbidden}{Encrypted content but no decryption key in reference}
\responseItem{408}{Request Timeout}{Timeout retrieving referenced manifest}
\responseItem{420}{Enhance your calm}{Recovery initiated but request timed out}\end{routeResponse}
\end{apiRoute}




\begin{apiRoute}{PUT}{/access/\param{root}/\param{pubkey}}{Add entry for pubkey to the  ACT referred in the root access manifest \ref{def:act-api}}{
}{ }

\begin{routeParameter} 
\routeParamItem{root}{hex string - reference to root access manifest}
\routeParamItem{pubkey}{hex string - public key of grantee}
\end{routeParameter}
\begin{routeResponse}{application/json}
\responseItem{201}{Created}{Reference to new manifest root in response body}
\responseItem{400}{Bad Request}{Address or public key not well formed}
\responseItem{401}{Unauthorized}{Permission denied: creating session key failed}
\responseItem{403}{Forbidden}{Encrypted content but no decryption key in reference}
\responseItem{408}{Request Timeout}{Timeout retrieving referenced manifest}
\responseItem{420}{Enhance your calm}{Recovery initiated but request timed out}
\end{routeResponse}
\end{apiRoute}


\begin{apiRoute}{DELETE}{/access/\param{root}/\param{pubkey}}{Remove entry for pubkey from ACT referred in the root access manifest, see \ref{def:act-api}}{
}{ }

\begin{routeParameter} 
\routeParamItem{root}{hex string - reference to root access manifest}
\routeParamItem{pubkey}{hex string - public key of grantee}
\end{routeParameter}
\begin{routeResponse}{application/json}
\responseItem{201}{Created}{Reference to new manifest root in response body}
\responseItem{400}{Bad Request}{Address or public key not well formed}
\responseItem{401}{Unauthorized}{Permission denied: creating session key failed}
\responseItem{403}{Forbidden}{Encrypted content but no decryption key in reference}
\responseItem{408}{Request Timeout}{Timeout retrieving referenced manifest}
\responseItem{420}{Enhance your calm}{Recovery initiated but request timed out}
\end{routeResponse}
\end{apiRoute}



% \subsection{Recovery\statusorange}\label{spec:api:recovery}

\section{Communications  \statusorange}\label{spec:api:communications}
\input{specs/api/comms.tex}

\subsection{PSS \statusyellow}\label{spec:api:trojan}

\begin{apiRoute}{POST}{/pss/send/\param{topic}(?targets=\param{targets}\&recipient=\param{recipient})}{Send private message with topic to targets, encrypted for recipient, see \ref{def:send}}{
}{ }

\begin{routeParameter} 
\routeParamItem{topic}{string}
\end{routeParameter}
\begin{queryParameter} 
\queryParamItem{recipient}{hex string - recipient public key for encryption}
\queryParamItem{targets}{hex string - comma separated list of targets}
\end{queryParameter} 
\begin{headerParameter} 
\headerParamItem{SWARM-TAG}{hex string}
\headerParamItem{SWARM-STAMP}{hex string}
\end{headerParameter}
\begin{routeResponse}{application/json}
\responseItem{209}{sent}{Tag to monitor}
\responseItem{400}{Bad Request}{Topic, targets or recipient not well formed.}
\end{routeResponse}
\end{apiRoute}




\begin{apiRoute}{POST}{/pss/subscribe/\param{topic}/(?on=\param{channel})}{Subscribe to messages with topic to be delivered on given channel, see \ref{def:receive}}{
}{ }

\begin{routeParameter} 
\routeParamItem{topic}{string}
\end{routeParameter}
\begin{queryParameter} 
\queryParamItem{on}{hex string - channel ID}
\end{queryParameter} \begin{routeResponse}{application/json}
\responseItem{201}{Created}{}
\responseItem{400}{Bad Request}{Topic. or channel not well formed.}
\end{routeResponse}
\end{apiRoute}

 
\begin{apiRoute}{DELETE}{/pss/subscribe/\param{topic}/(?on=\param{channel})}{Unsubscribe for topic on channel, see \ref{def:receive}}{
}{ }

\begin{routeParameter} 
\routeParamItem{topic}{string}
\end{routeParameter}
\begin{queryParameter} 
\queryParamItem{on}{hex string - channel ID}
\end{queryParameter} \begin{headerParameter} 
\headerParamItem{SWARM-TAG}{hex string}
\headerParamItem{SWARM-STAMP}{hex string}
\end{headerParameter}
\begin{routeResponse}{application/json}
\responseItem{204}{No content}{Successfully uninstall}
\responseItem{400}{Bad Request}{Topic. or channel not well formed.}
\end{routeResponse}
\end{apiRoute}


\subsection{Feeds \statusorange}\label{spec:api:feeds}



% \subsection{The pss URL scheme}

